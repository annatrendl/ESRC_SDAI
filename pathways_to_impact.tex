See the section https://je-s.rcuk.ac.uk/Handbook/Index.htm#pages/GuidanceonCompletingaStandardG/CaseforSupportandAttachments/ESRCSpecificRequirements.htm on Pathways to Impact for ESRC.



We will work with researchers at the College of Policing, the Behavioural Insights Team at West Midlands Police, 

Our day-to-day contact at West Midlands Police is Inspector Jenny Richards (Evidence Based Practice in Risk and Organisational Learning). Our day-to-day contact at the West Midlands Police and Crime Commissioner's Office is Alison Spence. 

At the level of the National Police Chiefs' Council we will work with Deputy Chief Constable Louisa Rolfe who is the lead for domestic abuse. 



In the second year of the proposal we will hold a 1-day impact summit at Warwick Business School's London Shard venue. We will host the National Police Chiefs' Council lead for Domestic Abuse, Deputy Chief Constable Louisa Rolfe (who is also at West Midlands Police), and Members of Parliament from the Home Affairs Committee domestic abuse inquiry. The summit will be a 50/50 academic / practitioner mix. We will contact, early in the project, [Insert list of charities in this sector] and visit them at their offices to present the programme of research and involve them as we conduct the initial analysis.

We will work with West Midlands Police to import the insight from the research into their day-to-day processes and also to their senior management team: 1. The programme on ``predictors of serious harm'' into the THRIVE+ (threat, harm, risk, investigation, vulnerability, engagement, pervention and intervention) risk assessment model used for assessing calls and reports of domestic abuse. 2. The programme on the  ``decision to report'' links with police's design of the inital contact with victims (e.g., https://west-midlands.police.uk/your-options/domestic-abuse) and ``police mis-recording of abuse'', which is of direct interest to their data quality and reporting. 3. The programme on ``envionmental factors influencing abuse'' can be used to help predict which local populations are most at risk of abuse in the West Midlands, 

At the College of Policing (situated just a few miles from Warwick campus) we will work within the framework of the Authorised Professional Practice on domestic abuse, released in 2015. (https://www.app.college.police.uk/app-content/major-investigation-and-public-protection/domestic-abuse/introduction/) and the revision of the existing DASH risk assessment tool for domestic abuse. (https://whatworks.college.police.uk/Research/Documents/DA_risk_assessment_pilot.pdf) We will work initially with David Tucker, Head Crime and Criminal Justice at the College of Policing.

Within Lloyds Bank, we will work with Martin King, Head of Customer Vulnerability. Warwick already has a relation with King's team in CI Stewart's work on the effects of gambling, which was completed using micro-level transaction data from millions of account holders. Key issues here are work on coercion and control, and having sufficient financial resiliance to escape an abusive relationship and being able to make the decision to report an abuser. Our day-to-day contact at Lloyds is Dr David Leake, Head of the Behavioural Insights Team. We will present the findings from all four programmes to the Lloyds Vulnerability Team in six-monthly briefings, and work to integrate insight from the research into the Lloyds environment. 


Events:

Impact summit (1 day at the Shard) \TM{We haven't budgetted anything for this...}
Visits to key charities
Visits to West Midlands Police
Presentation to NPCC
Data sprint session with WMP Behavioural Insights Team?


In addition to the academic journal outputs, we will present the findings in a series of thought leadership pieces for dissemination across the police and third-sector contacts. We will submit these reports to the College of Police What Works database and log them on the Research Map (https://whatworks.college.police.uk/Research/Research-Map/Pages/Research-Map.aspx)

Although a high-profile ``launch'' event at the Shard will attract attention, it is the continuous week-on-week contact with our impact partners that we think will deliver the legacy impact for this project.




PI Mullett will spend half of his time of the project leading the impact activity and takes overall responsibility of delivering the impact programme. Mullett has worked extensively on the impact programme from the ESRC funded ``Accumulating to Choose'' programme, with the Financial Conduct Authority, and in his research on predicting alcohol related crime or health crises. CI Stewart will also spend half of his time of the project on the impact activity. Stewart has significant experience of impact acitivites in the domain of financial decision making with Lloyds Banking Group, Barclays Bank, and the UK Finance Association. Postdoc Trendl will be fully involved in the impact programme, and this is also an important part of the professional development for an early career researcher, and will allow her to establish a network of independent contacts. Trendl will spend one third of her time on the programme on the impact activities.

We will apply to the Warwick ESRC Impact Acceleration Fund and to Warwick Business School Impact Fund to continue the impact programme beyond the end of the funded period. 





\section{Old Accumulating to Choose Pathways to Impact}

Pathways to Impact
We will target those in industry and public policy with interest in economic decision making.
While the focus of this programme is on informing micro-level theory of economic choice,
we will develop the implications for regulation and for the competitiveness of UK financial
industry. Our strategy will be to translate the laboratory work detailed in the Case for Support
into the field with a series of randomised control trials, detailed below. To deliver significant
impact it is essential for us to bridge the divide from the laboratory by working on live
problems for the regulator and industry using real consumers making real decisions.
Since their inception in 2012 the Financial Conduct Authority (the UK regulator for all
financial products, like savings accounts, pensions, mortgages, credit cards, and insurance)
have placed significant emphasis on behavioural regulation. This includes “...getting a better
understanding of why consumers act in the way they do, so we can adapt our regulation to
their [consumers’] common behavioural traits” (Martin Wheatley, outgoing Chief Executive,
FCA). Indeed, the first publication from the FCA made the case for using behavioural
economics in regulation (Erta, Hunt, Iscenko, Brambley, 2013). Recently Stewart has worked
closely with the new Behavioural Economics and Data Science Team at the FCA, led by Dr
Stefan Hunt, on the psychology of structured deposits (Hunt, Stewart, \& Zaliauskas, 2015).
The FCA are at the beginning of a project on the complexity of financial products. The
pathway to impact within the FCA will be via regular meetings between members of the FCA
Behavioural and Data Science Team, and Stewart and Mullett (most recently in August 2015).
Ultimately, as the data will drive public policy change regarding presentation and advertising
of financial products, with the creation of regulations grounded in behavioural science. The
outcome of this engagement will be the design of randomised control trials in collaboration
with the FCA and a customer-facing financial services provider using real consumers making
real decisions. Dissemination via the FCA working paper series will have more significant
reach than academic publication alone. Please see the letter of support from the Financial
Conduct Authority.
Warwick has a formal collaboration with the consumer group Which?, and Stewart has
worked closely with Alex Chesterfield, Head of Behavioural Insights, and team members
Sam Himmelweit, Harriet Patterson, and Phillida Cheetham. The formal collaboration covers
data sharing, exploitation, and publication rights, etc., and we have several existing and
completed projects (e.g., on credit card repayment behaviour and expectant mums’ birth
choices). Early insights from this work will feed into Which?’s project on the complexity of
financial products. They key route will be the monthly joint meetings between Warwick and
Which? Please see the letter of support from Which? Again, we would seek to implement
insight from these grants in a practical financial decision domain as part of Which?’s
consumer testing programme, and thus we will identify a practical application of the drift
diffusion model to a consumer choice of value to which and implement a field trial. Likely
candidates are: (a) the development of product information sheets for mortgages, savings
accounts, pensions, etc., within the context of providing behavioural nudges, and (b)
information provision for telecoms choices.
The UK Cards Association is the industry body for all credit card providers in the UK.
Stewart (and Gathergood, Economist, Nottingham) have a significant existing collaboration,
with the sharing of millions of credit card statements between about 10 of the largest card
companies and Warwick and Nottingham Universities. Contacts are Paul McCarron, Head of
Cards and Fraud Control, and Richard Koch, Head of Policy. Stewart contributed evidence to
the 2009 White Paper reviewing of the credit industry. Of key interest is (a) consumers’

decisions around credit card repayments, and their integration of information from monthly
statements and (b) consumers choice of card during switching and how cards are evaluated.
The former fits well with the evaluation work; the latter fits well with the choice work.
Dissemination of research findings will be via presentation to the Current Affairs working
group, with representatives from every UK card provider.
We will also work closely with the Economic and Social Research Institute, an independent
research institute funded mainly by the Irish Government. Our key contact is Dr Peter Lunn,
who has expertise in the application of decision making and consumer psychology research to
policy and regulation.
These routes to impact are time consuming. They would be led by the PI, with significant
involvement from the CI to distribute the load. We anticipate that 15% (i.e., half) of PI time
will be spent pursuing impact. These contacts will also add value for the PDRA and CI as
they can be taken to build research collaborations beyond the grant, independently of the PI.
We have requested funding to develop the field trials over the course of the grant, and in
previous experience the ability to make modest commitment to a larger programme of
randomised control trials enhances engagement.
Beyond these three existing routes to impact, during the course of the grant we will seek to
extend our contacts and influence to (a) one major UK supermarket, (b) one major UK
insurance company, and (c) one major bank.
Table 1. Translational field studies.
Translational Field Study

Anticipated Design

1. Financial Conduct
Authority

A representative sample of 2,000 investors in structured
deposits, payday loans, or other complicated financial
products.

2. Which?

Presenting multi-attribute choice options to 2,000
consumers switching broadband and or expectant mums.

3. UK Cards

A representative sample of 2,000 cardholders making (a)
direct debit setup decisions or (b) monthly repayment
decisions.

Finally, the opportunity to work with real consumers making real decisions with non-trivial
economic consequences (maybe tens of pounds with broadband, hundreds of pounds with
credit cards, and thousands of pounds with structured deposits of mortgages, and hours of
pain with expectant mums) is of significant academic value. The issue of incentivized choice
is critical in Experimental Economics (though less so in Psychology), and this impact work
extends the lab work into high stakes choice.

\section{Old Perfect Cop Pathways to Impact}


1

Pathways to Impact
This research is likely to deliver impact in recruitment of new officers, identifying outcomes
associated with the best management, reductions in the rates of sickness, reductions in
complaints and better responses to complaints, and matching officers to roles. In particular,
matching responses from our Online Lab to the police data adds value to those data. For
example, if, as we suspect, conscientiousness does indeed correlate with rates of sickness
and complaints, this would allow forces to understand sickness and complaints as, at least in
part, a function of individual differences in personality interacting with the local context in
which officers and staff are placed. And if well-being is associated with the quality of
management, this is a first step in developing mechanisms for improving the well-being of
officers and staff. Below we detail the pathways to these potential impacts.
The police support this research and have provided their data and access to officers and
staff without imposing constraints on the academic research programme. We have met with
and discussed the project with all of the officers and staff named here (and see meeting
dates on p. 2 of the Case for Support). At the MPS, this research project reports to the Total
Professionalism Board, which is chaired by our force lead Assistant Commissioner Helen
King. The Board is responsible workforce development and recruitment strategy. We have
already made presentations to the Board (April 2015, March 2016) and received their
backing. By integrating this project with the Board’s remit, we have official and long-term
support for the project. At Nottinghamshire the project reports to Chief Constable Sue Fish.
At States of Jersey the project reports to Chief Officer Mike Bowron.
The findings of this study have relevance to both metropolitan and county forces throughout
England, Wales and Northern Ireland (whom all operate under a common Doctrine and
Professional Standard). At the local level we will give at least six monthly feedback. For each
force, each Chief Constable and Police and Crime Commissioner pairing has independent
authority over their area (or MPS Commissioner / London Mayor pairing), and we will use
these meeting to develop interventions and analysis around the five themes below. At the
national level, the MPS is the largest police force in the UK and is a leader in practice. The
route to national-level impact is the National Police Chiefs’ Council (business area lead is
Chief Constable Giles York; Chief Constable Sue Fish is our NPCC sponsor) and through
the College of Policing who inform the National Approved Professional Practice (Rachel
Tuffin, Director, Knowledge, Research, and Education). At the international level, the high
regard of British policing provides a reference point for international standards. We have
already made links with the Chicago Citizens’ Police Data Project, Harvard Sociology, and
Chicago Police Department, and with New York Police Department via Warwick’s Institute for
Sustainable Cities. National and international level feedback will be via Warwick’s Centre for
Operational Policing Research, co-directed by PI Stewart with CI Hodgson.
Feedback to forces will be structured against the following themes, which have been
highlighted in discussions with senior officers, who will lead the implementation within their
forces. Feedback points are every six months, as agreed with the officers and staff named
below.
Public Desire. The research will deliver a list of psychological indicators of good
performance, reduced complaints and misconduct, and reduced physical and mental ill
health. These measures will inform the recruitment process and the selection of officers and
staff for specialist roles in the MPS. But changing recruitment processes represent a shift
from the implicit objective of police in the image of the public they serve. For example, “We
can’t have a police force representative of a London which has an ethnic minority population
of 55% when only 10% of its officers are from an ethnic minority” (Lammey, 13 January
2014; the 55% is probably a little too high here). Many forces use positive recruitment of
BME officers, and the MPS used residence requirements in 2013/14. How have recent
changes in recruitment changes the psychological attributes selected for? There is certainly
a political imperative for this directive, but little empirical evidence that the public wish this as

2
a determining factor for officer or staff recruitment. To this end, we will complement the
behavioural science evidence base with a survey of the public to measure the properties of
officers they desire (with Chief Constable Sue Fish, Nottinghamshire; Assistant
Commissioner Martin Hewitt, MPS; and the (London) Mayor’s Office for Police And Crime).
We will use a market research company to measure, from a representative sample of the UK
population, their views on the age, sex, race, ethnicity, religious affiliation, psychological
traits. Our objective is to start and develop a public dialogue.
Management. What is the impact of performance development review gradings on both
individual behaviours and those of their subordinates (Deputy Commissioner Craig Mackey
and Chief Superintendent Robert Jones). That is, what outcomes in are associated with
better line managers? We have already found that the highest scoring line managers have
subordinates with a 20% reduction in the incidence of sickness. We have also identified
significant variability across the Met within the sickness and complaints measures, with
some boroughs having half the rates of sickness and complaints of others, and some
boroughs having large changes within the 5-year period of the data. The mechanism for
impact within the MPS is the HR change programmes (Robin Wilkinson, HR Director).
Absenteeism and complaints. Using the data we already have, we see a very strong
association between sickness and complaints. Among officers who have no complaint in the
5-year period, 63% of officers have one or more episode of sickness. But among officers
who have a complaint, 84% have one of more sickness. Further, econometric panel data
analysis shows that there is almost no association between previous complaints and current
sickness or vice versa: Complaints do not appear to cause sickness, and sickness does not
put officers at greater risk of a complaint. The logical conclusion from this work is that there
are strong individual difference predictors of these measures—and we have
conscientiousness and agreeableness from the online lab as prime candidates. So key
questions are why are complaints and sickness associated (Chief Constable Sue Fish,
Nottinghamshire)? Which individual differences predict complaints and sickness (Chief
Superintendent Robert Jones)?
Right for the role? Do certain types of behavioural attributes (personality, risk attitudes)
coalesce in specific police functions, such as specialist operations or investigations (Chief
Officer Mike Bowron, States of Jersey Police)?
Bad cops or bad circumstances? Are there career hotspots for misconduct or serious
substantiated complaints (Deputy Commissioner Craig Mackey). Can we provide early
intervention warnings for officers at greatest risk of misconduct? The mechanism for impact
is through Professional Standards (Deputy Assistant Commissioner Fiona Taylor), to help
design career paths that avoid potentially potent placements or role succession that
materially increase the risk of misconduct.
We will also offer a series of four 1-day Behaviour Masterclasses to Senior Officers'
departments in the final year of the project. Classes will be advertised via the College of
Policing, the National Police Chief’s Council, and Warwick’s Centre for Operational Policing
Research. We will target the chief officers responsible for policy and implementation within
their force, and members of their teams. Classes will be co-led by Stewart or Ritchie, and an
officer. We will use feedback to shape the further development of the research.


