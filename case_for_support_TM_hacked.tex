\documentclass[11pt, a4paper]{article}

% Helvetica
\usepackage[scaled]{helvet}
\renewcommand\familydefault{\sfdefault}
\usepackage[T1]{fontenc}

% 2cm margins
\usepackage[margin=2cm]{geometry}

\usepackage{graphicx}
\usepackage{xcolor}
\usepackage[center]{caption}
\usepackage{hyperref}
\usepackage{placeins}
\usepackage{booktabs}
\usepackage{titling}
\usepackage{titlesec}
\setlength{\droptitle}{-7em}  
\usepackage{array}
\newcolumntype{L}[1]{>{\raggedright\let\newline\\\arraybackslash\hspace{0pt}}m{#1}}
\newcolumntype{C}[1]{>{\centering\let\newline\\\arraybackslash\hspace{0pt}}m{#1}}
\newcolumntype{R}[1]{>{\raggedleft\let\newline\\\arraybackslash\hspace{0pt}}m{#1}}

\usepackage{threeparttable}
%\newcolumntype{L}{>{\centering\arraybackslash}m{3cm}}


\usepackage{apacite}
%\usepackage{setspace}
%\doublespacing

\definecolor{green}{rgb}{0,1,0}
\newcommand{\NS}[1] {{\textcolor{green}{#1}}}
\newcommand{\TM}[1] {{\textcolor{orange}{#1}}}
\newcommand{\AT}[1] {{\textcolor{blue}{#1}}}

\posttitle{\par\end{center}\vskip 0.1em}
\titlespacing\section{0pt}{12pt plus 4pt minus 2pt}{4pt plus 2pt minus 2pt}

\begin{document}
\title{Understanding domestic abuse}
\date{}
\maketitle

\section{Introduction}

Domestic abuse is a complex phenomenon affecting people from all walks of life. It is increasingly recognised as a major public policy concern in many countries, including the UK \cite{ep}. While anyone can become a victim of domestic abuse, women are disproportionately affected, with more than 25\% of women, and 15\% of men in England and Wales reported to have experienced some form of domestic abuse since the age of 16 \cite{ONS}. In the most extreme cases, domestic abuse can culminate in domestic homicide: in the period between April, 2017 and March, 2018, 70 people in England and Wales were killed by their current or former partner, and 90\% of these victims were women \cite{homic}. Domestic abuse has substantial mental health implications, which are not limited to the direct target of abuse \cite{ferrari}, but can have a profound effect on the lives of children witnessing domestic abuse at home \cite{callaghan}. Reflecting the severity of the problem, the economic cost of domestic abuse in England and Wales between April, 2016 and March, 2017 was estimated to be as high as \pounds 66 billion, which includes the cost of missed workdays, health care costs, and the reduced quality of life for survivors \cite{costs}.


%The current legal definition in the UK aims to capture the multifaceted nature of domestic abuse, by recognising that domestic abuse encompasses a wide range of behaviours, including emotional, sexual, and physical abuse, threatening, intimidating, coercive and controlling behaviour \cite{govuk}.

%Domestic abuse has substantial mental health implications, with an estimated three-quarter of survivors experiencing posttraumatic stress disorder symptoms, and are significantly more likely to report symptoms of anxiety and depression compared to the general population \cite{ferrari}. The long lasting impacts of domestic abuse are not limited to the direct target of abuse. Witnessing domestic abuse at home can have severe developmental impacts on children, including an increased risk of mental and physical health problems, difficulties in interpersonal relationships in later life, worse educational attainment, and a higher likelihood of engaging in criminal behaviours \cite{callaghan}.

%As the legal definition reflects, one characteristic of domestic abuse that differentiates it from other violent crimes is its repeated nature. Estimates show that before getting effective help, on average, survivors live in the abusive relationship for 2.7 years, experiencing an estimated 50 cases of abuse \cite{SafeLives2015}. The most reliable statistics on domestic abuse in the UK is the Crime Survey for England and Wales (CSEW; \citeNP{ONS}), a victimisation survey, which includes a self-completion module on domestic abuse. According to the CSEW, of those respondents who experienced any form of domestic abuse between April 2017 and March 2018, only 17\% reported it to the police \cite{ONS}. This extremely high level of underreporting is another characteristic that is specific to domestic abuse compared to other types of crimes.


%In the most extreme cases, domestic abuse can culminate in domestic homicide. In the period between April, 2017 and March, 2018, 70 people in England and Wales were killed by their current or former partner, 90\% of these victims were women \cite{homic}, demonstrating that domestic abuse is a fundamentally gendered phenomenon. While the pervasive problem of underreporting poses a significant obstacle to deriving reliable estimates of the true extent of the problem, the economic cost of domestic abuse in England and Wales between April, 2016 and March, 2017 was estimated to be \pounds 66 billion \cite{costs}. The largest component of this cost is represented by the physical and emotional consequences of abuse, reflected in a reduced expected quality of life for survivors. In addition, lost economic output resulting from missed workdays and reduced productivity, as well as costs to the health care system also significantly contribute to the overall figure.

This research programme consists of four research topics aiming to deepen our understanding of domestic abuse victimisation, by exploiting the rich information contained in the CSEW, and combine it with insights from other data sources. First, we will explore the demographic and socio-economic characteristics of victims of domestic abuse and the predictors of suffering serious harm. Second, we will examine the reasons behind not just under-reporting from victims, but also police mis-recording of domestic abuse cases and victim's perception of the police after reporting an incident. Third, we will attempt to quantify the long-term effects of being a victim of domestic abuse and witnessing domestic abuse at home as a child. Finally, we explore the association between exogenous, time-varying factors, such as changes in alcohol consumption and financial resilience, and the prevalence of domestic abuse.

%Recognising the severity of this widespread societal problem led the UK government to develop a strategic plan to tackle domestic abuse with a funding of £100m for the period between 2016 and 2020. The action plan focuses on a range of areas to reduce the prevalence of domestic abuse, including increasing awareness and willingness to report through education and information campaigns, introducing legal measures to increase survivor safety (e.g., Domestic Violence Protection Order, Domestic Violence Disclosure Scheme, intervention programmes), and improve the responses of support services and health care professionals \cite{vawag}.

%These interventions are crucial in light of the fact that after a long decline, domestic violence against women has significantly increased between the period of 2008 and 2014 \textit{ref}.

%\newpage


\section{Data sources}

The research projects detailed below draw on various data sources. All of the projects rely on the CSEW, an annual, cross-sectional representative survey collecting information on victims of crime across England and Wales, including 30,000--40,000 households every year \cite{OfficeforNationalStatistics2019}. The survey has a self-completion module containing questions about the respondent's experiences of domestic abuse throughout their life (since the age of 16), and in the past 12 months in particular. Given the serious problem of underreporting, one of the unique characteristics of domestic abuse, the CSEW is the most reliable source of information on the prevalence of domestic abuse in England and Wales. We have access to this dataset via the UK Data Service.

%Some of the projects also benefit from a crime dataset provided by the West Midlands Police (WMP; the third largest police force in England, serving a population of 2.9 million; \citeNP{Homeoffice}), which includes all crimes and incidents recorded by the police force in the period between January, 2010 and October, 2018. The WMP is the third largest police force in England, serving a population of 2.9 million \cite{Homeoffice}. Crimes that have a domestic abuse marker indicate cases of domestic abuse that meet the criteria for notifiable offences in the UK (e.g. assault, or GBH), whereas domestic abuse incidents refer to cases that do not qualify as a crime. About 31\% of all crimes and incidents have a domestic abuse marker. For each record in this dataset, we have rich information about the circumstances of the case, including exact location and time of the incident or crime, the gender, age and ethnicity of the offender and victim, their address, and the severity of the injury sustained by the victim, if any. The first and last occurrence of the offence, as well as the exact time of reporting is also recorded. Each person in this dataset has a unique person identifier, allowing us to follow people over time. Access to this dataset is highly sensitive and strictly controlled. All researchers on this project have already been vetted by WMP and have been granted access to the required data fields. The PI, as data controller for the project is vetted to level 3, and the CoI's to level 2.


Some of the projects also benefit from an anonymised crime dataset provided by the West Midlands Police (WMP; the third largest police force in England, serving a population of 2.9 million; \citeNP{Homeoffice}), which includes all crimes and incidents recorded by the police force between 2010 and 2018, of which about 31\% are domestic abuse-related. For each record in this dataset, we know the exact location, time, the gender, age, ethnicity and address of those involved, and the severity of the injury. Each person has a person identifier, allowing us to follow people over time. Access to this dataset is highly sensitive and strictly controlled. All researchers on this project have already been vetted by WMP and have been granted access to the required data fields. The PI, as data controller for the project is vetted to level 3, and the CoI's to level 2.

These two datasets have different merits. The CSEW provides us with rich demographic and socio-economic information on the respondent, and to investigate the casual effects of domestic abuse victimisation, we can construct a control group from those individuals who have not experienced domestic abuse. However, it does not contain any information on the perpetrators and has limited longitudinal information. In contrast, the WMP dataset has information on both the victim and the offender, and records all reported incidents for the same victim-offender pair, allowing us to analyse temporal patterns and identify high-risk times of the year. We will not link the individuals in these datasets at any time.

%These two datasets have different merits, and combining them allows us to gain a deeper understanding the characteristics and dynamics of domestic abuse, and deliver policy-relevant insights. A unique benefit of the CSEW is that it allows us to investigate the causal effect of domestic abuse by providing information on a control group (those who have not suffered domestic abuse). In addition, the CSEW is more likely to give a better estimate of the true prevalence of domestic abuse, and contains rich demographic, socio-economic and geographic information on the respondent (survivor). However, owing to the fact that it is a victimisation survey, it does not contain any information on the perpetrators, and has limited longitudinal information. In contrast, the WMP crime dataset has information (age, gender, ethnicity) on both the victim and the offender, and records all reported incidents for the same victim-offender pair, allowing us to conduct a quantitative analysis of the dynamics of abusive relationships, and identify the predictors of escalation. Information on the exact timing of the abuse and the time of reporting will help us identify high-risk times of the year and understand the factors influencing the decision to report. Furthermore, the crime dataset provides us with information on other criminal behaviours of the offender. We will not link the individuals in these datasets at any time.


One proposed project relies on a dataset containing credit card and current account spending of 20 million customers across the UK. We have access to this data as part of a data sharing agreement with one of the ``Big 4''  UK banks. As we know which neighbourhood (LSOA) the card owners live in, this data will allow us to construct a neighbourhood-level measure of spending on alcohol, gambling, and benefit payments throughout time. At no point will individual data be used, only geographic, LSOA level, aggregates. In addition, for analyses that rely on neighbourhood-level characteristics, we will use the Indices of Multiple Deprivation (IMD; \citeNP{IMD}). The IMD ranks all 32,844 LSOAs in the UK based on their relative deprivation on various domains, including income, employment, education, health, crime, housing and living environment, providing us with a complex descriptor of the socio-economic characteristics of each neighbourhood in the country.



\section{Research Programme}


This research programme addresses four overarching topics, exploring different aspects of domestic abuse. Drawing on the extensive, rich respondent information in the CSEW and the unique, temporal nature of the WMP data, this research complements, extends and deepens our understanding of domestic abuse. In outlining the analysis plan for each of these research topics, we will specify a set of statistical analyses we plan to conduct. We will preregister all of our analysis plans and complement all analyses with robustness checks, and further exploratory regressions as the data allows (see, for an example of this approach in some of our \href{https://cog.warwick.ac.uk/~pssmar/Trendl_Stewart_Mullett.pdf}{prior work} on domestic abuse).

\subsection*{1. Exploring the characteristics of survivors and the predictors of serious harm}

%\NS{Add note at the end of each section on publication plan, suggesting specific journal titles, (e.g., PNAS, Nature Human Behaviour, Psychological Science, some domain specific journals.}

% \textcolor{red}{\textit{Short intro about aims, highlighting added value.}}
In this research topic, we will first use the CSEW to investigate the demographic and socio-economic characteristics of survivors of domestic abuse. Understanding the risk factors of domestic abuse victimisation is key for designing effectively targeted policy measures. Previous research has identified a number of risk factors predicting domestic abuse victimisation (e.g., being female, young, unemployed, separated or divorced, living in single parent household, earning in the lowest income bracket, having a long-term disability), based on descriptive statistics derived from the CSEW data \cite{ONS}. However, the CSEW provides a much broader range of information on the characteristics of the respondent, including number of children, ethnicity, education, benefit history, physical and mental health, house and car ownership, self-reported well-being, substance abuse, and frequency of going out. 

Investigating the explanatory power of a broader set of victim characteristics within one (logistic) regression model will provide us with a deeper understanding of the relative importance of these factors in predicting domestic abuse victimisation, compared to a purely descriptive statistics approach. We are also interested in how the predictive power of victim characteristics varies with the type of abuse suffered (physical abuse, threats, sexual abuse).

In addition, given the complex nature of domestic abuse, it is likely that there are important interactions between various victim characteristics in predicting the risk of domestic abuse victimisation (e.g., it is possible that not having stable employment is a more significant risk factor of domestic abuse victimisation for those with young children). To uncover the structure of these interactions without having to a priori specify them -- as is required in a logistic regression model -- we will also use a random forest classification algorithm, a machine learning method that has the ability to detect non-linear relationships between variables in predicting categorical outcomes (e.g., domestic abuse victimisation), allowing us to identify particularly vulnerable subgroups of the population \cite{rforest}. Insights from both methods will provide us with the first extensive characterisation of the risk factors predicting domestic abuse victimisation in England and Wales, using a national-level, large representative sample like the CSEW.

%\TM{Do we know enough about random forest classification to be confident about it's appropriateness? Could an expert reviewer find a reason to complain about it or criticize it's use? I ask because I don't know it well at all.}





%\textcolor{red}{\textit{Detailed plan}} First, using the CSEW data, we will extend our understanding of the demographic and socio-economic factors predicting domestic abuse victimisation. The CSEW provides a broad range of information on the characteristics of the respondent, including age, sex, marital status, number of children, ethnicity, education, employment, income, benefit history, physical and mental health, frequency of going out, house and car ownership, self-reported well-being, and frequency of drug and alcohol use. Previous research has identified a number of risk factors predicting domestic abuse victimisation (being female, young, unemployed, separated or divorced, living in single parent household and earning in the lowest income bracket), based on descriptive statistics derived from the CSEW data \cite{ONS}. Building on this prior work, we will explore the predictive power of a broader range of victim characteristics. Investigating the explanatory power of these characteristics within one statistical model will provide us with a deeper understanding of the relative importance of these factors in predicting domestic abuse victimisation, compared to a purely descriptive statistics approach. We are also interested in how the predictive power of victim characteristics vary with the type of abuse suffered (physical abuse, threats, sexual abuse).



 We will also explore the victim and offender characteristics that predict serious harm. Previous studies investigating the risk factors of serious harm in the context of domestic abuse mostly relied on police data, and focused on identifying high-risk offenders \cite{thornton}. Using a multinomial logistic approach, we will use the CSEW to investigate how victim characteristics predict the level of harm suffered, including financial independence and resilience (income bracket, employment stability), mental health (self-reported well-being), drug and alcohol dependence (frequency of usage), social isolation (self-reported frequency of going out, living alone or with the perpetrator, family disputes), recent separation, and feeling frightened (increased home security, reporting feeling scared).
 
 Using the WMP data, we will also explore the risk factors of serious harm associated with the perpetrator. The strength of the WMP data is that it has a temporal dimension, and we can follow the same perpetrator-victim pair over time. Using a logistic regression approach with perpetrator-victim fixed effects (FE), we will examine whether acts of serous harm can be predicted by previous self-harm and suicide attempts (which required police attendance), previous violent and drug offences, breaches of court orders, co-habitation with, and separation from the victim. Insights from these analyses will help the police to assess the validity of their current risk assessment model, and can inform decisions about the optimal timing and target audience for domestic abuse awareness campaigns to protect victims. 
 
 The findings from this work will result in high impact journal papers focussing upon the socio-economic risk factors of domestic abuse victimisation and the social and interpersonal relationship patterns predicting serious harm. The target journals for these works are the Journal of Personality and Social Psychology, Social Psychological and Personality Sciences, Journal of Research in Crime and Delinquency, and Personality and Social Psychology Bulletin.
 
% \AT{publication plan. Journal of Interpersonal Violence. Trauma, Violence, \& Abuse. American Journal of Public Health.}
 
% \TM{Social psychology or gender issues journals}


% Using the CSEW, we are interested in understanding how for each offence type (sexual, physical with injury, stalking and harassment), the police's response depend on the victim's socio-economic characteristics
%
% From the CSEW, we can estimate what percentage of reported cases that ended in warning, arrest, charge, or no action, a well as the victim's satisfaction with the police handling the case and their perception of their own security as a result of the outcome.
%
%https://www.justiceinspectorates.gov.uk/hmicfrs/our-work/article/crime-data-integrity/reports-rolling-programme-crime-data-integrity/




%Police forces in England, and particularly the West Midlands Police, have previously been criticised for mis-recording violent cases of violent crime, including \href{https://www.bbc.co.uk/news/uk-england-468676578}{domestic abuse.} Mis-recording these already heavily underreported crimes decrease victim's trust in the Criminal Justice System and their willingness to report subsequent incidents. \textit{Why don't we use the aggregate number of \href{https://www.ons.gov.uk/peoplepopulationandcommunity/crimeandjustice/datasets/domesticabuseinenglandandwalesappendixtables}{police-recorded domestic abuse cases for each police force} to do this? But if we don't particularly rely on the WMP data is there a point in this? They use \href{https://www.ons.gov.uk/peoplepopulationandcommunity/crimeandjustice/bulletins/domesticabuseinenglandandwales/yearendingmarch2018}{CSEW and police data} in this report, but they don't seem to compare the two}

\begin{table*}
\caption{Exploring the characteristics of survivors, and understanding the police response to reported cases, analysis plan}
  \begin{threeparttable}[t]
  \centering
       \begin{tabular}{ L{2.4cm}  C{1.4cm}  C{1.5cm}  C{3.6cm}  C{3.1cm}  C{2.6cm} }
    \toprule
     \textbf{Research question} & \textbf{Dataset}    & \textbf{Unit of analysis} & \textbf{Outcome variables} & \textbf{Explanatory variables} & \textbf{Model} \\
    \midrule
    What are the characteristics of domestic abuse victims? & CSEW & Individual  & Domestic abuse victimisation; Victimisation by type (physical, sexual abuse, threats) & demographic and socio-economic characteristics & Logistic regression/random forest \\
           \midrule
    What victim characteristics predict serious harm? & CSEW & Individual  & Level of harm resulting from domestic abuse victimisation (no physical harm, minor, serious) & financial resources, mental health, dependence, social isolation, separation, feeling scared & Multinomial logistic regression \\
                \midrule
                    What perpetrator characteristics predict serious harm? & WMP & Perpetrator-victim pairs & Violent offences resulting in injury & previous self-harm and suicide attempts, violent and drug offences, breach of court orders, separation &  Logistic regression with perpetrator-victim FE \\
%    Beznau II & 23.10.1971 & Pressurised water reactor & 365   & 2031 \\
%    Mühleberg & 01.07.1971 & Boiling water reactor & 373   & 2019\\
%    Gösgen & 02.02.1979 & Pressurised water reactor & 1010  & 2039 \\
%    Leibstadt & 24.05.1984 & Boiling water reactor & 1220  & 2044 \\
     \bottomrule
  \end{tabular}
%     \begin{tablenotes}
%     \item[1] Assumption: life time of 60 years.
%     \item[2] Official shut-down.
%   \end{tablenotes}
    \end{threeparttable}%
  \label{tab:addlabel}%
\end{table*}%

%Second, we can validate some of these findings using the WMP data, since the CSEW contains information on whether the survivor reported the domestic abuse to the police. The WMP contains information about the sex and age of victims. We can contrast the sex-age victimisation numbers from the WMP with corresponding population estimates to validate the findings of the CSEW \textit{(we were initially talking about using victims of burglary, but maybe this is better?)}. We can also complement this analysis with data on the offender's sex  and age, to estimate the prevalence of different forms of reported domestic abuse (child to parent violence, parent to child violence, intimate partner violence) and contrast it with estimates based on self-reported data from the CSEW, which distinguishes between family and intimate partner violence.

%As mentioned, domestic abuse is a complex phenomenon that encompasses a number of notifiable criminal offences that are usually repeat incidents in nature (e.g., violence against the person including physical or verbal abuse, sexual offences, criminal damage, stalking and harassment). However, the CSEW mainly focuses on




%\textbf{Aim 2.}  The CSEW is used to estimate the costs of domestic abuse in the UK. While a unique strength of the CSEW is that it provides reliable estimates of the proportion of the population who have been a survivor of a certain crime, it cannot capture the true extent of domestic abuse victimisation for several reasons. First, because it does not collect detailed information on various aspects of repeat domestic abuse victimisation (an inherent characteristic of domestic abuse cases), as it only records the number of incidents involving physical violence perpetrated by household members (and ignores repeated cases of non-physical violence). In addition, survivors who suffer frequent, ongoing abuse are less likely to accurately remember the exact number of incidents. Second, because it only focuses on certain sub-categories of domestic abuse (physical violence, sexual assault, stalking), while ignores others (criminal damage, verbal abuse, financial abuse, burglary). In addition, it does not collect information on whether different offences against the same survivor were perpetrated by the same offender. In contrast, the WMP data allows us to tell if a reported case was domestic abuse related (as the police are required to record this information), regardless of the actual offence committed. In addition, it allows us to identify various incidents with the same victim-offender pair, providing us with reliable data on reported cases of repeat victimisation. We can estimate the co-morbidity of domestic abuse victimisation with other types of criminal victimisation using the CSEW. We can combine this with the estimated likelihood of domestic abuse victimisation by offence type to get a better estimate of the overall extent of domestic abuse victimisation (reported and unreported). \textit{is this silly? then we can maybe estimate what percentage of true victimisation is domestic-abuse related? the £66 billion cost estimate is based on the CSEW}


%It is common that individuals are more likely to be a survivor of crime if they have previously been a survivor of crime, and if they have perpetrated crimes. It is also more likely that an individual will commit criminal activities if they have done so before, and if they have previously been the survivor of crimes. We will use our datasets to improve our understanding of the temporal patterns of victims and perpetrators, particularly around domestic abuse. The high level of detail in the CSEW allows us to examine the comorbidity of domestic abuse with other crimes during the previous year, and since the age of 16.


%By contrasting extrapolated estimates from the CSEW (which is a representative sample) with WMP data on the recorded number of reported cases, we can estimate the true extent of the mis-recording of domestic abuse cases by the WMP. Given the repeated cross-sectional nature of the CSEW, and the longitudinal nature of the WMP, we can quantify the proportion of mis-recorded incidents over time. We can complement this investigation with analysing the self-reported outcome of incidents that have been reported to the police from the CSEW, and the survivor's perception of the CJS.



%co-occurrence of other types of crimes - we can contrast results from the CSEW with results from our crime data - e.g., if we see that those reporting domestic abuse have also experienced criminal damage in the CSEW, we can check in the WMP data if criminal damage or home theft is usually a spillover effect from domestic abuse ; deliberate violence; account theft



%We will use the CSEW to understand the demographic and socioeconomic predictors of the decision to report the abuse to the police, and identify specific groups who are less likely to report. We are able to do this because CSEW respondents indicated whether they reported the abuse to the police  and whether the police came to know of the abuse in another way (Question X).
%
%Regression 1: data --- subset of people who in CSEW responding that they have been abused
%
%	abuse or not ~
%   	[demographics] age + sex + marital status + earnings + ...
%	other crime dummies + ...
%
%
%	unreported abuse / reported abuse ~
%	[demographics] age + sex + marital status + earnings + ...
%	other crime dummies + ...
%	[information about abuse] level of harm + ...


%Respondents also answered questions indicating the level of harm that they have suffered, which will allow us to estimate how reporting and police awareness depend upon the level of harm. \NS{Anna: broaden this paragraph to be about all of the covariates}

%Linear probability model (and logit model?)

%We can contrast and validate these findings using the WMP dataset.
%The WMP data are cases that have been reported to the police. The CSEW data allow an estimate of the true underlying prevalence of domestic abuse. The discrepancy between the two datasets allows us to estimate the reporting gap.
%That is, we can use the WMP dataset to provide a second estimate of the degree of underreporting, indpendent of the CSEW respondent's indications of report. Discrepancy between reporting prevelence estimated from CSEW and the number of cases in the WMP data give important information about the police's recording and categorisation of domestic abuse cases---do police flag all of the cases of domestic abuse?  https://www.bbc.co.uk/news/uk-england-46867657

%While the CSEW provides us with accurate information on the characteristics of victims, it does not contain information on the perpetrators. Combining the two datasets helps us find commonalities between the perpetrators of abuse cases that are.


%Domestic abuse doesn't discriminate---well we can estimate this by comparing the 7.7\% for females and 4.4\% for males in the past year with the size of the coefficients.

%We can describe who is abused and their demographics in CSEW, CSEW who report, and in WMP.

%We can also compare those abused in WMP data with any other reference group in the WMP data. (e.g., burgled) and then use DA-burgled in CSEW vs WMP to back out stuff.


%\textcolor{red}{\textit{Core outcomes}}

%The first extensive investigation of the demographic and socio-economic predictors of domestic abuse victimisation. A deeper understanding of the characteristics of police mis-recording and its effect on victim's trust in the CJS.




\subsection*{2. Understanding the decision to report and police mis-recording of domestic abuse}

In this research topic, we will first explore the factors influencing victim's decision to report the abuse. Even though improving willingness to report is key to effectively tackle domestic abuse, we are not aware of previous research exploring this question in depth. Using the CSEW and logistic regression approach, we will investigate how the decision to report (or seek external help) depends on the individual-level, time-invariant characteristics of the survivor, including trust in the justice system, social ties (living with family, time living in the area, member of Neighbourhood Watch), financial independence (own income, employment status, car ownership), number of children, mental health, and alcohol and drug dependency. 

We will complement this analysis with the exploration of time-varying factors affecting the decision to report ongoing abuse using the WMP. We are interested in whether these reports are more likely to occur after certain days of the year (e.g., birthday of the victim or perpetrator, Christmas, New Years, Valentine's Day etc.). This is particularly important knowledge for police, so that they can efficiently target awareness campaigns to further encourage reporting.
 
 
%  the victim and offender characteristics that predict serious harm. Identifying high-risk victim-offender pairs is key in preventing serious harm. Insights from these analyses can inform decisions about the optimal timing and target audience for domestic abuse awareness campaigns.
%
%\textcolor{red}{\textit{What we know so far}}
%
%\href{https://www.thehotline.org/2018/09/28/escalation/}{Escalation} often happens when the abuser feels like they are losing control (e.g., because the survivor threatens to leave); separation is an important risk factors of serious harm; contact over children
%
%\href{http://www.safelives.org.uk/policy-evidence/about-domestic-abuse/how-long-do-people-live-domestic-abuse-and-when-do-they-get}{Reporting}
%
%decision to report: serious violence; friends, family, neighbours notice; happens in public, in front of people


We will also use the CSEW to explore the police's response to reported cases of domestic abuse. Increasing victim's trust in the Criminal Justice System (CJS) through improving law enforcement response to reported incidents is key for improving willingness to report. Mis-recording of domestic abuse is a serious problem across police forces, for example, a recent inspection reported that WMP have failed to record 25\% of reported crimes that were domestic abuse-related \cite{HerMajestysInspectorateofConstabularyandFires&RescueServices2019}. In this research topic, we will explore whether the extent of crime mis-recording depends on victim characteristics. Using the CSEW, we will be able to tell whether the type of domestic abuse reported by the respondent is likely to amount to a crime according to the list of notifiable offences \cite{countingrules}. A multinomial regression approach will allow us to examine the extent to which police mis-recording is different across broad crime types (sexual, physical violence, harassment and stalking), and depends upon the socio-economic characteristics of the victim and police forces (e.g., funding per police officer, deprivation). Complementing this analysis with a random forest approach will reveal the role of interactions between individual and police force characteristics. Lastly, these analyses can also be applied to examine the effect upon victim satisfaction with the police, and their perception of their personal security after reporting a crime.

We envisage two journal articles as a result of this work. One will focus upon the existing academic literature on decision making, and behaviour change, as this research will give unique insight into the process and factors that lead an individual to report. The target journals will be Nature Human Behaviour, and Psychological Science. The other paper will be more specifically focussed on the implications for policing, and this will be targeted towards the Journal of Research in Crime and Delinquency, and Criminology.

%\AT{publication plan. behavioural \& public policy journals?}

%Second, using the WMP data, we can analyse the temporal patterns of reported domestic abuse incidents by following  23\% of the 182,646 victim-offender pairs over an eight-year-long period.


\begin{table}[!htbp]
\caption{Understanding the decision to report and police mis-recording of domestic abuse, analysis plan}
  \begin{threeparttable}[t]
  \centering
       \begin{tabular}{ L{3cm}  C{1.1cm}  C{1.5cm}  C{3cm}  C{3.3cm}  C{2.8cm} }
    \toprule
     \textbf{Research question} & \textbf{Dataset}    & \textbf{Unit of analysis} & \textbf{Outcome variables} & \textbf{Explanatory variables} & \textbf{Model} \\
                \midrule
     What victim characteristics predict the decision to report? & CSEW & Individual  & Reported abuse to the police & type of abuse, trust in CJS, social ties, financial status, children, health, dependencies & Logistic regression \\
       \midrule
          What time-varying factors predict the decision to report? & WMP & Perpetrator-victim pairs & Reported ongoing abuse to the police & time of year, holidays, length of abuse, severity of last occurrence & Logistic regression \\                    
                     \midrule
    How does the police response depend on victim characteristics? & CSEW  & Individual & Police action (no action, warning, arrest, charge) &  socio-economic characteristics of victim and police force  & Multinomial logistic regression/ random forest \\
     \midrule
    What predicts victims' trust in the police? & CSEW  & Individual & Satisfaction with police action, perception of security & police action,  socio-economic characteristics of victim and police force & Multinomial logistic regression/ random forest \\
     \bottomrule
  \end{tabular}
    \end{threeparttable}%
  \label{tab:addlabel}%
\end{table}%


%We can examine whether a recorded survivor of domestic abuse was previously recorded as a survivor in other crimes, and if they were subsequently a survivor of other criminal behaviours. Identifying risk factors is crucial for police services who want to identify individuals at risk of being future victims, so that they can target interventions and give the best advice possible to reduce an individual's risk. In addition, we can measure the relationship between experiencing domestic abuse, and the likelihood of being recorded as a suspect in other crimes. It is possible that some criminal behaviour is a precursor to experiencing domestic abuse whilst others follow it.


%nd the factors determining the decision to report the abuse

%\TM{This feels out of place. Should the mis-recording and underreporting be in the same section? Maybe it would make more sense having analyses 3\&4 from this section actual put in section 1? If not, then we need to make it clearer what is the distinction.}


%Aim:
%The CSEW respondents indicated the level of harm, and we can use the CSEW to estimate the cross-sectional association between the most serious harm and other comorbid factors like alcohol, other crimes,

%Regression 3: type of abuse ~
%	visit hospital ~
%	mental health ~

%Separation is one of the biggest predictors of abuse resulting in serious injury. \textit{Subject to whether we can get data on the DVPOs (Domestic Violence Protection Order) and DVPNs (Domestic Violence Protection Notice) issued in the West Midlands in the relevant period, we could explore the offender characteristics that predict the likelihood of sustained perpetration after separation, and assess the overall efficacy of these new preventive measures. If not, we can still use info on location + breaches of non-molestation orders?}

%But cross-sectional data present a particular problem for making casual claims: For example, incidents involving drugs or alcohol intoxication: some individuals may end up in an abusive relationship because of an existing dependency, or a dependency may develop as a counterproductive strategy of coping with the abusive relationship.



%Should you be identifying the couples most at risk of serious harm from time-invariant properties, or should you be following couples over time and using time-varying properties to spot when to intervene?

%The WMP dataset will further allow us to follow victim-offender pairs over time, and explore how the severity and nature of reported abuse changes over time.
%In the WMP data we can follow X victim-perp pairs for X years, with each pair appearing X times on average in the data.

%Regression 4: offense classification[proxy for harm] ~ .
%	type of abuse ~ .
%
%Future survivor behaviours. Future perpetrator behaviours
%
%serious event at t ~ stuff at t-1  + perpetrator previous crimes
% and 	a model with pair fixed effects
%
%Murders are very rare, so we are adopting a ``near miss'' airline / nuclear industry strategy of predicting the more prevalent but less serious incidents.
%
%Say pair fixed effects control for all non-time varying characteristics. With and without pair fixed effects let's use see how much state vs trait matters for prediction.


\subsection*{3. The long-lasting effects of domestic abuse}

 Domestic abuse has long-lasting adverse effects on victims and children witnessing the abuse. In this research topic, our aim is to explore some of these consequences. The 2016 CSEW included a module asking respondents about whether they witnessed domestic abuse at home as a child. Descriptive statistics of this dataset revealed that those who reported to have witnessed abuse as a child were significantly more likely to report long-term health problems, and live in a single parent household as an adult \cite{ONSChildhood}. First, we will consolidate and extend these findings using propensity score matching, a statistical approach that allows for causal inferences to be drawn about the effect of witnessing abuse, whilst also controlling for a range of confounds, such as economic deprivation. 
 %Gaining a deeper understanding of the tangible, far-reaching consequences of domestic abuse will help to design survivor support programmes and quantify the societal harm caused by it.


%\textcolor{red}{\textit{What we know so far}}
%\href{https://www.womensaid.org.uk/information-support/what-is-domestic-abuse/the-nature-and-impact-of-domestic-abuse/}{Women's aid, effect on victims} increased likelihood of mental health problems (isolation, low self-esteem), alcohol and drug dependency, financial consequences due to financial abuse or losing a job as a result of the abuse, fighting back?
%
%\href{https://uksaysnomore.org/effects-of-domestic-abuse-on-children/}{Effect on children I} witnessing violence is major predictor for PTSD and depressive symptoms in children; behavioural problems, bullying (either victims or perpetrators)
%
%\href{https://www.psychologytoday.com/gb/blog/progress-notes/201902/alarming-effects-childrens-exposure-domestic-violence}{Effect on children II} cognitive development is affected, learning difficulties are more common, difficulties in social relationships, withdrawal, antisocial behaviours, they are more likely to become both perpetrators and victims of domestic abuse; protective factors: literacy, intelligence, social competence, supportive relationship with at least one influential adult



Second, we will explore how children's behavioural problems are affected by witnessing domestic abuse at home. The CSEW questionnaire for 10-15 year old's records information about experiences with bullying, carrying knives, gang membership, school truancy, learning difficulties and health outcomes, including drug use and drinking behaviour. By applying propensity score matching, we will explore the causal effect of living in an abusive household on these outcomes.

We will further complement this analysis using WMP data, by identifying whether younger offenders' home addresses can be linked to a past domestic abuse incident. It is not possible to predict whether witnessing domestic abuse increases the chance of becoming a young offender, but we can examine whether it predicts the type of crimes they engage in. For example, whether, after controlling for the socio-economic characteristics of the neighbourhood (using the IMD), such individuals are more likely to commit similarly violent acts than other young offenders.


 These analyses will constitute the first UK-based comprehensive quantitative exploration of the causal effect of domestic abuse and childhood abuse on socio-economic outcomes. The resulting paper will be targeted at The Lancet Public Health, and Social Science and Medicine.
 
 
% \TM{Developmental Psychology Journals?}
% \AT{publication plan. The Lancet Public Health. Annual Review of Public Health. American Journal of Public Health.}



\begin{table}[!htbp]
\caption{The long-lasting effects of domestic abuse, analysis plan}
  \begin{threeparttable}[t]
  \centering
       \begin{tabular}{ L{3.9cm}  C{1cm}  C{1.9cm}  C{3.2cm}  C{2.8cm}  C{1.6cm} }
    \toprule
     \textbf{Research question} & \textbf{Dataset}    & \textbf{Unit of analysis} & \textbf{Outcome variables} & \textbf{Explanatory variables} & \textbf{Model} \\
    \midrule
    How does witnessing domestic abuse as a child affect socio-economic outcomes in adulthood? & CSEW & Individual & Education, employment, health, domestic abuse victimisation & Witnessing domestic abuse & Propensity score matching \\
                \midrule
                   How does experiences of past domestic abuse affect socio-economic outcomes in adulthood? & CSEW & Individual & Employment, health & Degree and type of domestic abuse suffered & Propensity score matching \\
                \midrule
     How does living a household with domestic abuse affect behavioural outcomes in childhood? & CSEW & Individual & Bullying, gangs, school truancy, carrying knives, drugs and alcohol, learning difficulties, health & Degree and type of domestic abuse in household & Propensity score matching \\
                \midrule
          Are young offenders from abusive households more violent? & WMP, IMD & Young offenders & Type of offence (property-related, public order offence, violent) & Neighbourhood socio-economic characteristics, abusive household & Multinomial logistic regression \\
     \bottomrule
  \end{tabular}
    \end{threeparttable}%
  \label{tab:addlabel}%
%  \setlength\belowcaptionskip{-20pt}
\end{table}%
%\raggedbottom
%Using data from the CSEW and well as population data of the West Midlands, we can estimate how many children and young people below 18 in each LSOA of the West Midlands are at risk of witnessing domestic abuse at home each year.

%\textit{The problem is that we don't know the address of the young offenders. Could we the location info on where they committed the offences as a proxy of the LSOA where they live? not sure if this is viable; defining a gang member is problematic in the crime data - if something serious is committed probably only one of them will be charged with it, maybe we should just concentrate on criminal activities; I was trying to identify children who were victims of something happening at an address where DA happened and perpetrated by a DA offender - it's very narrow and excludes children who *just* witnessed abuse; however, the location id doesn't seem to specific to one place - one id can correspond to a road, dwelling, shop etc.}


%We can use school exclusion data - unfortunately this data is only available on the local authority and school-type (state-funded primary, secondary, special) level...

%There's a children's questionnaire, and they ask about gang membership. However, the sample size is very low (3,100 for the whole country).

%of the 119,539 under 18s who were victims of crimes (domestic/child abuse, or other more serious physical/sexual violence) in the dataset, 11\% were later offenders with an average of 2 offences (mostly domestic abuse, physical violence, drug possession and theft) - these are only the kids we have evidence were traumatised at some point

%of the 149,759 da offenders, 19\% have an incident against an under 18





%Aim:
%
%CSEW regression: child gang membership ~ any domestic abuse or reported DA [2x2 table of abuse/no abuse by gang / no gang]
%Also estimate, given gang membership what fraction are DA---which links ot WMP analysis below
%
%It also contains information about  multiple members of the household. This allows us to examine whether young individuals living in a household with a survivor of domestic abuse are more likely to be victims or perpetrators of other types of crime. By using other questions in the CSEW, we can also measure whether these children are more likely to perpetrate crimes, and to be involved in gang related activities.
%
% explore the lasting effect of domestic abuse on the victims and their children (do we see kids having behavioural problems? No control group unfortunately)- CSEW question about gang membership - are those who report having a child gang member more likely to be victims of domestic abuse? Following people over time (Dynamic topic modelling for victims)- what’s the temporal order?
%
%Match on name, location, and date of birth in WMP data to see effect on children
%
%WMP regression:
%
%take every juvinile in a gang and try to match all to domestic abuse cases----estimates of what fraction of gang members have co-morbid domestic abuse
%
%What fraction of reported child abuse has comorbid DA; what fraction of DA has comorbid child abuse
%
%Using school census we can go futher and estimate frequencies in a 2x2 child abuse  by domestic abuse table---providing we can match children to mothers.


\subsection*{4. Environmental factors influencing the prevalence of domestic abuse}

In this research topic, we will investigate some of the environmental predictors of domestic abuse. First, using the CSEW and a logistic regression approach, we will explore the neighbourhood predictors of domestic abuse victimisation and willingness to report, using information on the interviewer's perception of the street (signs of rubbish, vandalism, general condition of houses etc.), and neighbourhood deprivation (measured by the IMD), whilst controlling for the socio-economic characteristics of the respondent. 


These analyses will extend our understanding of the neighbourhood-level predictors of domestic abuse victimisation, and complement our individual-level approach outlined in the first project to provide a comprehensive exploration of the range of factors affecting domestic abuse victimisation and willingness to report.


%Additionally, we will investigate how neighbourhood (LSOA) deprivation (measured by the IMD) affects the prevalence of domestic abuse and willingness to report, using a spatial regression framework. 





%\textcolor{red}{\textit{What we know so far}} interestingly, there are no studies about this using UK data; neighbourhood-level predictors of domestic abuse: social disorganization theory (socio-economic disadvantage and residential instability disrupt social bonds a limit collective ability to maintain control and recognise IPV); cultural norms through social learning process (high IPV neighbourhoods where people observe influential others being rewarded for engaging in IPV, whereas in low IPV neighbourhoods perpetrators get socially ostracized); women's empowerment reduces IPV, especially financial independence;  abuser's lack of employment and alcohol consumption can be a risk factor; alcohol outlet density; churches, playgrounds, community networks might foster community cohesion and reduces IPV;  however, in deprived neighbourhoods, where IPV is accepted social interactions can increase IPV
%
%\href{https://www.sciencedirect.com/science/article/abs/pii/S0277953616302891}{The relationship between electronic gaming machine accessibility and police-recorded domestic violence: A spatio-temporal analysis of 654 postcodes in Victoria, Australia, 2005–2014}
%
%\href{https://journals.sagepub.com/doi/abs/10.1177/1525107115623938}{Economic Stress and Domestic Violence: Examining the Impact of Mortgage Foreclosures on Incidents Reported to the Police }
%
%\href{https://journals.sagepub.com/doi/abs/10.1177/1524838012445641?journalCode=tvaa}{
%The Impact of Neighborhoods on Intimate Partner Violence and Victimization
%}
%
%the type sof social tie matters: friendship ties reduce the risk of IPV, family ties have no effect
%
%\href{https://journals.sagepub.com/doi/abs/10.1177/1524838013515758}{Neighborhood Environment and Intimate Partner Violence: A Systematic Review }

%Using the CSEW and the transaction dataset, we will also explore how neighbourhood-level spending on alcohol and gambling predict the prevalence of domestic abuse in the area, whilst controlling for the deprivation of the neighbourhood. We will do so by estimating a spending profile on alcohol and gambling for each LSOA in a year. 



We will also explore the extent to which changes in alcohol consumption and financial stability on the community level affects the prevalence of domestic abuse. To do so, we will use the transaction data to construct an LSOA-level spending profile for a given time period to investigate how temporal changes in this profile affect the number of reported domestic abuse cases in that area, using a spatial panel regression model. This statistical estimation approach will alleviate the problem of spatial dependency between neighbourhoods, and will allow us to derive a precise estimate of the effect of temporal changes in community-level alcohol consumption and financial health on the prevalence of domestic abuse in the neighbourhood. We will identify external events that may affect alcohol consumption  and spending on gambling (e.g., festivals, weather, bank holidays, sport tournaments), and explore how these affect the reported number of domestic abuse incidents in that area in a given time period. Previously, our research has identified a 60\% increase in alcohol-related domestic abuse when the England national football team won.


In addition, drawing on previous findings about the link between financial stress and domestic abuse, we will explore what effect the roll-out of the Universal Credit (UC) across the West Midlands had on the reported number of domestic abuse cases. These reforms had been criticised widely, due to the temporal financial strain it imposes on the least financially resilient people in society. The fact that the exact date of the roll-out varied across the seven metropolitan boroughs within the county allows for a more precise estimation of the effect of UC on the reported number of domestic abuse incidents in the West Midlands. These analyses will constitute the first empirical investigation of the direct effect of welfare policy changes on domestic abuse victimisation. 




\begin{table}[!htbp]
\caption{Environmental factors influencing the prevalence of domestic abuse, analysis plan}
  \begin{threeparttable}[t]
  \centering
       \begin{tabular}{ L{4cm}  C{1.2cm}  C{1.6cm}  C{3cm}  C{2.8cm}  C{1.8cm} }
    \toprule
     \textbf{Research question} & \textbf{Dataset}    & \textbf{Unit of analysis} & \textbf{Outcome variables} & \textbf{Explanatory variables} & \textbf{Model} \\
    \midrule
    How do neighbourhood characteristics predict domestic abuse? & CSEW & Individual & domestic abuse victimisation/willingness to report & Perception of neighbourhood, socio-economic characteristics & Logistic regression \\
%         \midrule
%          How do community-level spending on alcohol and gambling affect domestic abuse? & CSEW, Transaction data, IMD & LSOA-level & domestic abuse victimisation & Spending on alcohol and gambling, deprivation & Spatial Poisson regression \\
         \midrule
    How do community-level temporal changes in alcohol consumption and income affect domestic abuse? & WMP, Transaction data & LSOA, day-, week-, month-level & Number of reported domestic abuse cases & Alcohol, gambling spending, benefits & Spatial panel Poisson/negative binomial regression \\

     \bottomrule
  \end{tabular}
    \end{threeparttable}%
  \label{tab:addlabel}%
\end{table}%

We envisage two articles. One will be focussed upon general behavioural theory and the effect of the environment. This will be targeted towards Nature Human Behaviour, and Proceedings of the National Academy of Sciences. The other article will focus upon the implications for criminology and policing. The target journal will be Criminology. 

% \AT{publication plan. fancy behav journals? Nature Human Behaviour, PNAS}

\newpage

%Unit of analysis is large geographical area in a year
%Time span 2005-
%
%DA or not ~
%
%Given DA: report or not ~
%
%Fixed effects for region and year
%
%matched in socieoconomic
%
%
%
%WMP analysis
%
%Unit of analysis is LSOA (MSOA?) per month or per quarter
%Time span - 2010-2020
%
%Number of DA cases ~

%The CSEW allows us to identify the demographic and socio-economic characteristics of victims who are the least likely to report, and therefore should be in the focus of support services. We can validate and contrast these findings with insights from a spatial regression analysis of the crime data, using low-level geographical area information from the census.

%Extend regression model X to include region fixed effects?
%
%
%\textit{A sentence about what earlier investigations have found about particular risk factors, including socio-economic deprivation.}

%deprivation is a risk factor for women
%
%Australia study on domestic abuse and proximity to fixed-odds betting terminals, and the one on offlicences.
%
%anything that affects propensity to drink/financial stress
%
%neighbourhood characteristics (deprivation), fixed odds betting terminals, alcohol outlets, events (football, festivals)
%
%\href{https://www.birminghammail.co.uk/news/midlands-news/universal-credit-rolled-out-across-14142901}{Universal credit roll-out in the West Midlands}
%
%\href{https://www.birminghammail.co.uk/black-country/universal-credit-claimants-black-country-15946831}{Universal credit roll-out in the Black Country}
%
%
%food bank dependence? data on housing e.g. evictions? homelessness.
%
%Mention we can construct some of these variables by access to "a very large UK bank" which whom we have a data sharing agreement. We cna build them from the row-per-transactino data for spending on current account and credit cards.

\clearpage

\bibliographystyle{apacite}
\bibliography{domesticabuse_refs}

\end{document}
