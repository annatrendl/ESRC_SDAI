\documentclass[11pt, a4paper]{article}

% Helvetica
\usepackage[scaled]{helvet}
\renewcommand\familydefault{\sfdefault} 
\usepackage[T1]{fontenc}

% 2cm margins
\usepackage[margin=2cm]{geometry}

\usepackage{graphicx}
\usepackage{xcolor}
\usepackage[center]{caption}
\usepackage{hyperref}
\usepackage{placeins}
\usepackage{booktabs}
\usepackage{array}
\newcolumntype{L}[1]{>{\raggedright\let\newline\\\arraybackslash\hspace{0pt}}m{#1}}
\newcolumntype{C}[1]{>{\centering\let\newline\\\arraybackslash\hspace{0pt}}m{#1}}
\newcolumntype{R}[1]{>{\raggedleft\let\newline\\\arraybackslash\hspace{0pt}}m{#1}}

\usepackage{threeparttable}
%\newcolumntype{L}{>{\centering\arraybackslash}m{3cm}}


\usepackage{apacite} 
%\usepackage{setspace}
%\doublespacing

\definecolor{green}{rgb}{0,1,0}
\newcommand{\NS}[1] {{\textcolor{green}{#1}}}
\newcommand{\TM}[1] {{\textcolor{orange}{#1}}}
\newcommand{\AT}[1] {{\textcolor{blue}{#1}}}

\begin{document}
\title{ESRC SDAI Application - Understanding domestic abuse using Big Data}
\date{}
\maketitle
\section{Objectives}

%Objectives
%
%The objectives of the proposed project should be listed in order of priority and should be those that the Investigators would wish the Council to use as the basis for evaluation of work upon completion of any grant awarded.
%
%This field must be completed using:
%
%    No more than 4000 characters, including spaces, tabs and character returns (2 characters).
%    Only the standard Je-S character set should be used.
%    No specialist characters and symbols (e.g. mathematical symbols) because these may not transfer successfully to other computer systems.
%    Please note that typing into a text box is not detectable by the system and is regarded as system inactivity. Please remember to save text regularly.
%

\textit{List the main objectives of the proposed research [up to 4000 chars]}

Who are the people most vulnerable to becoming victims of domestic abuse? To what extent do neighbourhood characteristics predict the prevalence of domestic abuse? What predicts serious harm and victims' decision to finally report the abuse? What is the quantifiable, long-term impact of domestic abuse victimisation, or witnessing domestic abuse as a child on socio-economic outcomes in later life? How do changes in alcohol consumption and financial stability affect the prevalence of domestic abuse? In this research project, we will use a wide range of quantitative approaches, and various data sources to answer these questions.

We have 3 core objectives. 

1) To conduct the first large-scale, quantitative exploration of the individual- and neighbourhood-level risk factors of domestic abuse victimisation in England and Wales. Through the rigorous validation of insights from existing, mostly quantitative investigations using an UK-wide dataset, we will provide a deeper understanding of the environments perpetuating domestic abuse in England and Wales. While we will only use UK-specific data, insights from these analyses will likely to be applicable to non-UK contexts. 

2) To improve our understanding of the costs of domestic abuse, by quantifying the causal effect of victimisation on a variety of socio-economic outcomes. Current cost estimates by the government do not take into account how educational attainment, income and employment status of victims are affected by victimisation and witnessing abuse as a child \cite{costs}. We will improve the precision of this calculation with the first, comprehensive UK-based exploration of the long-term effects of domestic abuse victimisation.

3) To translate our research into impact by delivering applicable insights for the relevant agencies, including the police and policy-makers. Our research will provide the police with important information to help the optimal geographical and temporal allocation of resources and improving recording practices. In addition, these insights will also help the government to design more effective intervention campaigns, and will be of interest to a wide range of academics.

4) To utilise the rich information contained in the Crime Survey for England and Wales (CSEW) to explore the predictors and consequences of domestic abuse victimisation. In spite of the detailed information contained in the CSEW about respondents, we are not aware of any non-exploratory study using the CSEW to explore the predictors and consequences of domestic abuse victimisation in England and Wales. We aim to take full advantage of the wealth of information in the CSEW in understanding domestic abuse.  

   
\section{Summary}


%Summary
%
%Note: This Summary will be published on publicly available sites should the project be funded. Please ensure confidential information is not included in this Summary.
%
%The Summary text box must be completed:
%
%    Maximum of 4000 characters (including spaces and returns).
%
%    Only the standard Je-S character set should be used.
%
%    No specialist characters and symbols (e.g. mathematical symbols) because these may not transfer successfully to other computer systems.
%
%The Summary is used for:
%
%    Sending to potential Peer reviewers to determine whether the proposal is within their field of expertise
%
%    To publicise the Councils research programmes to a variety of readers eg Opinion-formers and policy makers, the general public and the wider research community.
%
% Explain in plain English:
%
%    The context of the research.
%
%    Its aims and objectives.
%
%    Its potential applications and benefits.
%
%Please note that typing into a text box is not detectable by the system and is regarded as system inactivity. Please remember to save text regularly.

\textit{Describe the proposed research in simple terms in a way that could be publicised to a general audience [up to 4000 chars]}

Concern about domestic abuse is widespread, and it is rightly identified as a serious societal issue affecting the lives of people from all walks of life. However, the true extent of the problem is still often underestimated, and hard to comprehend: 2 million adults in England and Wales (7.9\% of women and 4.2\% of men) reported to have experienced some form of domestic abuse in the year ending 2018 March \cite{ONS}, and 41\% of all women murdered in England and Wales in the same period were victims of domestic homicide. The long-lasting, damaging effects of domestic abuse are not limited to the direct target of abuse, but also has serious developmental implications for children witnessing the abuse. Illustrating the extent of the societal harm caused by domestic abuse, the economic cost of domestic abuse in England and Wales between April, 2016 and March, 2017 was estimated to be as high as \pounds 66 billion, which primarily reflects the cost of missed work days, health costs, and the reduced quality of life for survivors \cite{costs}.


%Addressing these questions is fundamental for a deeper understanding of the far-reaching consequences of domestic abuse victimisation and the individual and environmental factors precipitating it. This knowledge can in turn form the basis of effective policy interventions aiming to encourage willingness to report and safeguarding groups vulnerable to domestic abuse. In this research programme, we aim to provide a comprehensive answer to these (and other) questions related to domestic abuse, with a series of quantitative analyses drawing on various data sources. 

%It is estimated that in the year ending 2018 March, 2 million adults in England and Wales (7.9\% of women and 4.2\% of men) have experienced some form of domestic abuse \cite{ONS}. Domestic abuse can result in homicide, and the risk of lethal violence is especially high in the case of female victims: 41\% of femicides, and 0.6\% of homicides committed in the year ending 2018 March were domestic abuse-related \cite{homic}. 

Despite the extent and severity of the problem, there is relatively little academic research focusing on domestic abuse in the context of the UK, perhaps due to the lack of reliable data sources. Using police data to understand domestic abuse is problematic due to the vast levels of underreporting, while insights from qualitative approaches are hard to generalise. In the UK, the most reliable source of data on domestic abuse is the Crime Survey for England and Wales (CSEW), an annual, cross-sectional representative survey collecting information on socio-economic characteristics as well as experiences of domestic abuse \cite{OfficeforNationalStatistics2019}.  While the Office for National Statistics (ONS) regularly publishes findings from the CSEW, these are mostly exploratory and descriptive in nature. In spite of the rich information provided by this survey, only a limited number of studies have utilised the CSEW to explore the predictors of domestic abuse victimisation in England and Wales (e.g., \citeNP{Khalifeh2013a}). 

%Domestic abuse is a complex, multifaceted phenomenon, that has a profound impact on the lives of those affected. It is increasingly recognised as a serious public health concern in many countries, including the UK. In the year ending 2018 March, an estimated 2 million adults in England and Wales (7.9\% of women and 4.2\% of men) have experienced some form of domestic abuse \cite{ONS}. Besides being more likely to experience any form of domestic abuse, compared to men, women are also significantly more likely to endure severe, often lethal violence at home: 41\% of femicides, and 0.6\% of homicides committed in the year ending 2018 March were domestic abuse-related \cite{homic}. The long-lasting effect of domestic abuse are not limited to the direct target of abuse, as it can have severe developmental impacts on children witnessing the abuse. Taking into account the emotional and physical consequences of the abuse suffered, the economic cost of domestic abuse in England and Wales between April, 2016 and March, 2017 was estimated to be \pounds 66 billion \cite{costs}. 

%Despite the vast economic costs associated with it, there is relatively little academic research focusing on domestic abuse in the context of the UK. This is most likely due to the scarcity of data sources relating to this topic. Owing to the fact that it occurs in private settings, domestic abuse is a vastly underreported crime: it is estimated that only 17\% of victims who experienced any form of domestic abuse between April 2017 and March 2018 reported it to the police \cite{ONS}, which means that studies exclusively drawing on police data to understand various aspects of domestic abuse (e.g. \citeNP{Bland2015}) have limited ability to derive useful insights about this phenomenon. Another methodological approach is to conduct interviews with survivors (e.g. \citeNP{callaghan}), which, while are tremendously helpful in illuminating the contexts in which domestic abuse can occur, c.

Our project fills this gap in our current understanding of domestic abuse by taking advantage of the wealth of information contained in the CSEW, and drawing on other data sources to complement the findings. Applying sophisticated statistical techniques tailored to each analysis (including regression, random forest, propensity score matching), we will identify the individual and neighbourhood-level predictors of domestic abuse victimisation, and the factors affecting victims' willingness to contact the police. We will also explore whether police mis-recording of domestic abuse-related crimes \cite{HerMajestysInspectorateofConstabularyandFires&RescueServices2019}  disproportionately affects certain victim groups. Furthermore, we will quantify the short- and long-term causal effects of witnessing domestic abuse on behavioural outcomes in young adulthood (with particular focus on the propensity to engage in violent criminal behaviours) and socio-economic and health-related outcomes in later life. Finally, we will explore how alcohol consumption, spending on gambling, and changes in benefit receipts affect the prevalence of domestic abuse.
 
 

Our work will primarily have implications for the police and policy-makers. For example, insights about the predictors of serious harm can inform the police protocol aiming to prevent domestic homicides, while understanding the extent to which crime mis-recording depends on victim characteristics will help the police understanding the causes of mis-recording and improve their practices. Policy-makers will be able to design more effective awareness campaigns using information on the geographical and temporal characteristics of domestic abuse victimisation, reporting behaviour, the extent to which gambling and alcohol spending affects it, and better understand its far-reaching societal implications, with particular focus on the effect of domestic abuse victimisation on propensity to violence in young adulthood and  socio-economic outcomes in later life. 

%64 characters over the limit - it will be different now (TM)

\section{Academic Beneficiaries}

%Please summarise how your proposed research will contribute to knowledge, both within the UK and globally.
%
%Academic Beneficiaries (4000 characters max) should address the following questions
%
%    how the research will benefit other researchers in the field
%
%    identify whether there are any academic beneficiaries in other disciplines and, if so, how they will benefit and what will be done to ensure that they benefit
%
%What other researchers, both within the UK and elsewhere are likely to be interested in or to benefit from the proposed research.
%
%Please look broadly beyond narrow research field. UKRI recognises that in generating new knowledge, a cross-disciplinary or single-disciplinary approach may be the most appropriate. Applicants are asked to clearly state their chosen approach and provide justification for that choice.
%
%List any academic beneficiaries from the research and give details of how they will benefit and how the results of the proposed research will be disseminated. Specific beneficiaries might be:
%
%    researchers in the investigator’s immediate professional circle, carrying out similar or related research
%
%    researchers in other disciplines
%
%    researchers in other academic institutions (in the UK and/or international)
%
%Also describe the relevance of the research to beneficiaries:
%
%    identify the potential academic impact of the proposed work
%
%    show how the research will benefit other researchers (this might include methodological or theoretical advances)
%
%    identify whether the research will produce data or materials of benefit to other researchers, and explain how these will be stored, maintained and made available
%
%    explain any collaboration with other researchers and their role in the project
%
%Note: The Academic Beneficiaries section may be published to demonstrate the impact of Research Council funded research.  Please ensure confidential information is not included in this section.

\textit{Describe who will benefit from the research [up to 4000 chars].}

Owing to the variety of questions we aim to answer the diverse methodological approaches we will employ in addressing these questions, our research will deliver insights relevant to researchers across various social science disciplines, including psychology, behavioural scientists, and health economics. \TM{Do we want to mention criminologists as well or are we trying to avoid having them as reviewers?}

Psychologists working on understanding the risk factors of domestic abuse victimisation will benefit from the first quantitative, comprehensive, UK-based exploration of the risk factors of domestic abuse victimisation. Previous investigations focusing on domestic abuse in the context of the UK have either used police records \TM{Aren't we relying on police records? Can we make it clearer why we have a particular strength over those previous approaches?} or relied on a small-sample, qualitative approach.

Behavioural researchers will benefit from our work exploring how exogenous, time-varying factors affect the willingness to report, and the causal effect of alcohol consumption and financial shocks on the prevalence of domestic abuse. These insights will be particularly important for behavioural researchers working for Behavioural Insights Teams and the UK government, owing to their policy-relevant nature.


Health economists will be interested in our work on how direct or indirect domestic abuse victimisation affects socio-economic outcomes and physical and mental well-being in later life. These estimates will help us improve the precision of the estimated cost of domestic abuse in the UK.


\section{Staff Duties}

As PI, Mullett will be responsible for coordinating the research. He will also act as the data controller for this project. Mullett has attained level three vetting with West Midlands Police, allowing him more comprehensive access to original police records, and he has the appropriate experience coordinating large datasets within projects. He will act as supervisor to the early career researcher Anna Trendl, ensuring that she can continue to build upon the impressive research capabilities she has already developed during her PhD. As a member of the Centre for Operational Policing Research (COPR), he will, in tandem with Prof Neil Stewart, liaise with West Midlands Police to ensure that the results are communicated back in the most effective and interpretable manner. This will maximise the impact that this research produces.

As Co-I, Stewart's responsibilities will be to aid in coordinating the research program. He will also meet regularly with the early career researcher Anna Trendl, providing insights and support. A major role will also be to coordinate the communication of the results within the broader context of the Centre for Operational Policing Research (COPR) activities. This coordination, and the possibility for links and collaboration with other COPR members or projects will provide unique added value to the research. 

As Post-Doctoral researcher, Trendl will be responsible for performing the analyses upon the data, and leading the writing of the resulting papers.

\textit{Summarise the roles and responsibilities of each post for which funding is sought [up to 2000 characters]}

\section{Impact Summary }

%Impact Summary
%
%Note: If the proposed research is funded the Impact Summary will be published on publicly available sites to demonstrate potential impact of Research Council funded research.  Please ensure confidential information is not included in this Summary.
%
%The Summary text box must be completed:
%
%    · Maximum of 4000 characters (including spaces and returns)
%
%    · Only the standard Je-S character set should be used. See https://jes.rcuk.ac.uk/Handbook/pages/Generaldocumentactions/JeSstandardCharacterSet.htm.
%
%    · No specialist characters and symbols (e.g. mathematical symbols) because these may not transfer successfully to other computer systems.
%
%The Impact Summary should answer the two questions:
%
%    ·  Who might benefit from this research?
%
%    ·  How might they benefit from this research?



\textit{(please refer to the help for guidance on what to consider when completing this section) [up to 4000 chars]}

1) characteristics of victims and understanding predictors of serious harm - police can use insights to prevent serious harm, characteristics of victims - expand on previous qualitative research (academics) and help policy-makers better target campaigns

2) reporting behaviour and police mis-recording - police can improve recording practices, understanding reporting behaviour can benefit behavioural scientists who work for the government

3) consequences - helps the government and academics by advancing our knowledge on the consequences of domestic abuse victimisation by improving the methodology of
  
  
  

\section{Ethical Information}

\textit{Please explain what, if any, ethical issues you believe are relevant to the proposed research project, and which ethical approvals have been obtained, or will be sought if the project is funded? If you believe that an ethics review is not necessary, please explain your view (available: 4000 characters)}


\section{Attachments}

\href{https://je-s.rcuk.ac.uk/Handbook/Index.htm#pages/GuidanceonCompletingaStandardG/CaseforSupportandAttachments/ESRCSpecificRequirements.htm}{Instructions for attachments}

* Case for support (6 page limit, 11 points, A4, 1" margins) 

* References

* Pathways to Impact (2 sides of A4) 

* Justification of Resources (2 sides of A4)

* Letter of support from WMP

* Data management plan (modify from \href{https://github.com/neil-stewart/data_management_plan}{lab plan}


* Accumulating to choose progress report

* NIBS2 progress report



\newpage

\bibliographystyle{apacite}
\bibliography{domesticabuse_refs}

\end{document}