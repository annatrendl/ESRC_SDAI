\documentclass[11pt, a4paper]{article}

% Helvetica
\usepackage[scaled]{helvet}
\renewcommand\familydefault{\sfdefault} 
\usepackage[T1]{fontenc}

% 2cm margins
\usepackage[margin=2cm]{geometry}

\usepackage{graphicx}
\usepackage{xcolor}
\usepackage[center]{caption}
\usepackage{hyperref}
\usepackage{placeins}
\usepackage{booktabs}
\usepackage{array}
\newcolumntype{L}[1]{>{\raggedright\let\newline\\\arraybackslash\hspace{0pt}}m{#1}}
\newcolumntype{C}[1]{>{\centering\let\newline\\\arraybackslash\hspace{0pt}}m{#1}}
\newcolumntype{R}[1]{>{\raggedleft\let\newline\\\arraybackslash\hspace{0pt}}m{#1}}

\usepackage{threeparttable}
%\newcolumntype{L}{>{\centering\arraybackslash}m{3cm}}


\usepackage{apacite} 
%\usepackage{setspace}
%\doublespacing

\definecolor{green}{rgb}{0,1,0}
\newcommand{\NS}[1] {{\textcolor{green}{#1}}}
\newcommand{\TM}[1] {{\textcolor{orange}{#1}}}
\newcommand{\AT}[1] {{\textcolor{blue}{#1}}}

\begin{document}
\title{ESRC SDAI Application - Understanding domestic abuse using Big Data}
\date{}
\maketitle
\section{Objectives}
\textit{List the main objectives of the proposed research [up to 4000 chars]}


1) To conduct the first large-scale, quantitative exploration of the individual- and neighbourhood-level risk factors of domestic abuse victimisation. Rigorous validation of insights from existing, mostly quantitative investigations using an UK-wide dataset, providing a deeper understanding of the environments perpetuating domestic abuse. While we will only use UK data, these findings will likely to be relevant for researchers outside the UK. 

2) To translate our research into impact by delivering applicable insights for the relevant agencies.  Police (improving recording practices, understanding domestic abuse reporting demand characteristics over time for a better allocation of resources), government (temporal/geographical planning of targeted campaigns, intervention programmes), shelters (temporal/geographical allocation of resources), academia (publications)
 
\section{Summary}

\textit{Describe the proposed research in simple terms in a way that could be publicised to a general audience [up to 4000 chars]}

\textit{I'm really bad at this. I don't really understand what should go in here}

Concern about domestic abuse is widespread, and it is rightly identified as a serious issue. However, the true severity is still often underestimated, and hard to comprehend: Of all women murdered in the UK, 41% die as a direct result of domestic abuse. 

% As a result, we collect all this data at great expense, but it is massively underused in quantitative research.

% Our analyses are more advanced than existing descriptives available from ONS etc.

% Then put specific outputs and uses of the results e.g.

% Police need to know where to direct resources, we can tell them what individuals and communities are most at risk

% There has been little quantitative work identifying the risk factors for most serious harms, including murder. We will identify the relative risk factors. This should directly link back to the opening sentences.

% Police want to run campaigns to encourage reporting and help lift victims out of a cycle of violence and suffering, but they don't know the best time or targets for these awareness campaigns. We can show what are the triggers for reporting long term abuse.



Who are the people most vulnerable to becoming victims of domestic abuse? To what extent do neighbourhood characteristics predict the prevalence of domestic abuse? What predicts serious harm and victims' decision to finally report the abuse? What is the quantifiable, long-term impact of domestic abuse victimisation, or witnessing domestic abuse as a child on socio-economic outcomes in later life? How do changes in alcohol consumption and financial stability affect the prevalence of domestic abuse?

Addressing these questions is fundamental for a deeper understanding of the far-reaching consequences of domestic abuse victimisation and the individual and environmental factors precipitating it. This knowledge can in turn form the basis of effective policy interventions aiming to encourage willingness to report and safeguarding groups vulnerable to domestic abuse. In this research programme, we aim to provide a comprehensive answer to these (and other) questions related to domestic abuse, with a series of quantitative analyses drawing on various data sources. 

It is estimated that in the year ending 2018 March, 2 million adults in England and Wales (7.9\% of women and 4.2\% of men) have experienced some form of domestic abuse \cite{ONS}. Domestic abuse can result in homicide, and the risk of lethal violence is especially high in the case of female victims: 41\% of femicides, and 0.6\% of homicides committed in the year ending 2018 March were domestic abuse-related \cite{homic}. The long-lasting effect of domestic abuse are not limited to the direct target of abuse, as it can have severe developmental impacts on children witnessing the abuse. The economic cost of domestic abuse in England and Wales between April, 2016 and March, 2017 was estimated to be \pounds 66 billion \cite{costs}. 

Despite the extent of the problem, there is relatively little academic research focusing on domestic abuse in the context of the UK, perhaps due to the lack of data sources. Using police data is problematic due to the vast levels of underreporting, while insights from qualitative approaches are hard to generalise. In the UK, the most reliable source of data on domestic abuse is the Crime Survey for England and Wales (CSEW), an annual, cross-sectional representative survey collecting information on experiences of domestic abuse \cite{OfficeforNationalStatistics2019}.  While the Office for National Statistics (ONS) regularly publishes findings from the CSEW, these are mostly exploratory and descriptive in nature. In spite of the rich information provided by this survey, only a limited number of studies have utilised the CSEW (e.g., \citeNP{Khalifeh2013a}) to understand domestic abuse. 

%Domestic abuse is a complex, multifaceted phenomenon, that has a profound impact on the lives of those affected. It is increasingly recognised as a serious public health concern in many countries, including the UK. In the year ending 2018 March, an estimated 2 million adults in England and Wales (7.9\% of women and 4.2\% of men) have experienced some form of domestic abuse \cite{ONS}. Besides being more likely to experience any form of domestic abuse, compared to men, women are also significantly more likely to endure severe, often lethal violence at home: 41\% of femicides, and 0.6\% of homicides committed in the year ending 2018 March were domestic abuse-related \cite{homic}. The long-lasting effect of domestic abuse are not limited to the direct target of abuse, as it can have severe developmental impacts on children witnessing the abuse. Taking into account the emotional and physical consequences of the abuse suffered, the economic cost of domestic abuse in England and Wales between April, 2016 and March, 2017 was estimated to be \pounds 66 billion \cite{costs}. 

%Despite the vast economic costs associated with it, there is relatively little academic research focusing on domestic abuse in the context of the UK. This is most likely due to the scarcity of data sources relating to this topic. Owing to the fact that it occurs in private settings, domestic abuse is a vastly underreported crime: it is estimated that only 17\% of victims who experienced any form of domestic abuse between April 2017 and March 2018 reported it to the police \cite{ONS}, which means that studies exclusively drawing on police data to understand various aspects of domestic abuse (e.g. \citeNP{Bland2015}) have limited ability to derive useful insights about this phenomenon. Another methodological approach is to conduct interviews with survivors (e.g. \citeNP{callaghan}), which, while are tremendously helpful in illuminating the contexts in which domestic abuse can occur, c.

Our project takes advantage of the rich information contained in the CSEW, and draws on other data sources to characterise domestic abuse in England and Wales. We will identify the individual and neighbourhood-level predictors of domestic abuse victimisation, and the factors affecting victims' willingness to contact the police. We will also explore whether there is any evidence that the police's response depend on the individual victim characteristics. We will quantify the causal effect of domestic abuse on the behavioural outcomes of child and the socio-economic outcomes of adult victims. Finally, we will explore how alcohol consumption, spending on gambling, and changes in benefit receipts affect the prevalence of domestic abuse.
 
 
para on implications (500 chars)
Our work will have implications for the police and policy-makers. The police will benefit from insights regarding the individual- and neighbourhood-level characteristics of particularly vulnerable victim groups (in terms of domestic abuse victimisation, reporting behaviour, and mis-recording of reported crimes), predictors of serious cases, and times of the year when the number of reported cases is likely to be higher. Policy-makers will be able to design more effective awareness campaigns using information on the geographical and temporal characteristics of domestic abuse victimisation, and better understand its far-reaching societal implications (effect on socio-economic characteristics, knife crime).  



\section{Academic Beneficiaries}

\textit{Describe who will benefit from the research [up to 4000 chars].}

\section{Staff Duties}

\textit{Summarise the roles and responsibilities of each post for which funding is sought [up to 2000 characters]}

\section{Impact Summary }

\textit{(please refer to the help for guidance on what to consider when completing this section) [up to 4000 chars]}

\section{Ethical Information}

\textit{Please explain what, if any, ethical issues you believe are relevant to the proposed research project, and which ethical approvals have been obtained, or will be sought if the project is funded? If you believe that an ethics review is not necessary, please explain your view (available: 4000 characters)}


\section{Attachments}

\href{https://je-s.rcuk.ac.uk/Handbook/Index.htm#pages/GuidanceonCompletingaStandardG/CaseforSupportandAttachments/ESRCSpecificRequirements.htm}{Instructions for attachments}

* Case for support (6 page limit, 11 points, A4, 1" margins) 

* References

* Pathways to Impact (2 sides of A4) 

* Justification of Resources (2 sides of A4)

* Letter of support from WMP

* Data management plan (modify from \href{https://github.com/neil-stewart/data_management_plan}{lab plan}


* Accumulating to choose progress report

* NIBS2 progress report



\newpage

\bibliographystyle{apacite}
\bibliography{domesticabuse_refs}

\end{document}