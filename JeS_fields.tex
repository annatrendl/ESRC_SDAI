\documentclass[11pt, a4paper]{article}

% Helvetica
\usepackage[scaled]{helvet}
\renewcommand\familydefault{\sfdefault} 
\usepackage[T1]{fontenc}

% 2cm margins
\usepackage[margin=2cm]{geometry}

\usepackage{graphicx}
\usepackage{xcolor}
\usepackage[center]{caption}
\usepackage{hyperref}
\usepackage{placeins}
\usepackage{booktabs}
\usepackage{array}
\newcolumntype{L}[1]{>{\raggedright\let\newline\\\arraybackslash\hspace{0pt}}m{#1}}
\newcolumntype{C}[1]{>{\centering\let\newline\\\arraybackslash\hspace{0pt}}m{#1}}
\newcolumntype{R}[1]{>{\raggedleft\let\newline\\\arraybackslash\hspace{0pt}}m{#1}}

\usepackage{threeparttable}
%\newcolumntype{L}{>{\centering\arraybackslash}m{3cm}}


\usepackage{apacite} 
%\usepackage{setspace}
%\doublespacing

\definecolor{green}{rgb}{0,1,0}
\newcommand{\NS}[1] {{\textcolor{green}{#1}}}
\newcommand{\TM}[1] {{\textcolor{orange}{#1}}}
\newcommand{\AT}[1] {{\textcolor{blue}{#1}}}

\begin{document}
\title{ESRC SDAI Application - Understanding domestic abuse using Big Data}
\date{}
\maketitle
\section{Objectives}
\textit{List the main objectives of the proposed research [up to 4000 chars]}
\section{Summary}

\textit{Describe the proposed research in simple terms in a way that could be publicised to a general audience [up to 4000 chars]}

\textit{Not sure what the right angle is, my current approach is to 1) demonstrate the extent of the problem 2) highlight the gaps in our current knowledge and 3) very briefly explain how are we contributing}

Domestic abuse is a complex, multifaceted phenomenon, that has a profound impact on the lives of those affected. Anyone can become a victim, regardless of socio-economic status. The official definition of domestic abuse in the UK aims a wide range of behaviours, including emotional, sexual, and physical abuse, threatening, intimidating, coercive and controlling behaviour \cite{govuk}. 


Domestic abuse is increasingly recognised as a serious public health concern in many countries, including the UK. In the year ending 2018 March, an estimated 2 million adults in England and Wales (7.9\% of women and 4.2\% of men) have experienced some form of domestic abuse \cite{ONS}. Besides being more likely to experience any form of domestic abuse, compared to men, women are also significantly more likely to endure severe, often lethal violence at home: 41\% of femicides, and 0.6\% of homicides committed in the year ending 2018 March were domestic abuse-related \cite{homic}. The long-lasting effect of domestic abuse are not limited to the direct target of abuse, as it can have severe developmental impacts on children witnessing the abuse. Taking into account the emotional and physical consequences of the abuse suffered, the economic cost of domestic abuse in England and Wales between April, 2016 and March, 2017 was estimated to be \pounds 66 billion \cite{costs}. 

Despite the vast economic costs associated with it, there is relatively little academic research focusing on domestic abuse in the context of the UK. This is most likely due to the scarcity of data sources relating to this topic. Owing to the fact that it occurs in private settings, domestic abuse is a vastly underreported crime: it is estimated that only 17\% of victims who experienced any form of domestic abuse between April 2017 and March 2018 reported it to the police \cite{ONS}, which means that studies exclusively drawing on police data to understand various aspects of domestic abuse (e.g. \citeNP{Bland2015}) have limited ability to derive useful insights about this phenomenon. Another methodological approach is to conduct interviews with survivors (e.g. \citeNP{callaghan}), which, while are tremendously helpful in illuminating the contexts in which domestic abuse can occur, are unfortunately limited in the generalisability of their findings.

In the UK, the most reliable source of data on domestic abuse is the Crime Survey for England and Wales (CSEW), an annual, cross-sectional representative victimisation survey across England and Wales, collecting data from 30,000--40,000 households every year \cite{OfficeforNationalStatistics2019}. The survey has a self-completion module containing questions about the respondent's experiences of domestic abuse throughout their life (since the age of 16), and in the past 12 months in particular.  While the Office for National Statistics (ONS) regularly publishes findings from the CSEW, these are mostly exploratory and descriptive in nature. In spite of the rich information provided by this survey, a limited number of studies have utilised the CSEW (e.g., \citeNP{Khalifeh2013a}) to understand domestic abuse. 
 
 By recognising its potential, we will rely on the CSEW and combine in with other external datasets to examine various aspects on domestic abuse in England and Wales, and provide a better understanding of this phenomenon. We will explore the individual- and environmental-level risk factors of domestic abuse, including the demographic and socio-economic predictors of victimisation, as well as the neighbourhood characteristics affecting it. Drawing on previous reports on police mis-recording of domestic abuse across England, we will explore whether this affects specific victim groups disproportionately, and its consequences regarding victim's trust in the Criminal Justice System. By combining the rich information in the CSEW with a police crime dataset provided by the West Midlands Police, we will explore the victim- and perpetrator-specific predictors of domestic abuse resulting serious harm and the time-varying and time-invariant predictors of reporting the abuse. The CSEW will also allow us to estimate the causal effect of direct and indirect domestic abuse victimisation on socio-economic and health-related outcomes in adulthood and the likelihood of engaging in violent criminal behaviours in childhood. In addition, we will explore how time-varying external factors, such as changes in alcohol consumption and financial health affect the reported number of domestic abuse incidents. Our previous investigation have revealed that the reported number of alcohol-related domestic abuse increases by 61\% following the victory of the England national football team, highlighting the profound effect external events can have on perpetrators' propensity to engage in violence.





\section{Academic Beneficiaries}

\textit{Describe who will benefit from the research [up to 4000 chars].}

\section{Staff Duties}

\textit{Summarise the roles and responsibilities of each post for which funding is sought [up to 2000 characters]}

\section{Impact Summary }

\textit{(please refer to the help for guidance on what to consider when completing this section) [up to 4000 chars]}

\section{Ethical Information}

\textit{Please explain what, if any, ethical issues you believe are relevant to the proposed research project, and which ethical approvals have been obtained, or will be sought if the project is funded? If you believe that an ethics review is not necessary, please explain your view (available: 4000 characters)}


\section{Attachments}

\href{https://je-s.rcuk.ac.uk/Handbook/Index.htm#pages/GuidanceonCompletingaStandardG/CaseforSupportandAttachments/ESRCSpecificRequirements.htm}{Instructions for attachments}

* Case for support (6 page limit, 11 points, A4, 1" margins) 

* References

* Pathways to Impact (2 sides of A4) 

* Justification of Resources (2 sides of A4)

* Letter of support from WMP

* Data management plan (modify from \href{https://github.com/neil-stewart/data_management_plan}{lab plan}


* Accumulating to choose progress report

* NIBS2 progress report



\newpage

\bibliographystyle{apacite}
\bibliography{domesticabuse_refs}

\end{document}