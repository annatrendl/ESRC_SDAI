Justification of Resources

Staff, Directly Incurred Posts

PDRA Mullett. We request 100\% support for PDRA Mullett. A postdoctoral appointment is
essential here, as the level of technical skill required to run and analyse eye movement data is
high. As outlined in Staff Duties, Mullett will design the studies with the PI and lead on the
writing of the majority. Mullett will program the eye movement system and implement the
statistical analysis designed with the PI using the R and Matlab programming languages.
Mullett will also take a secondary role in the impact programme, and lead on the supermarket
collaboration.

PDRA at Caltech. We seek 100\% support for a 1-year PDRA at Caltech. The experimental
games eye tracking from Series 1, 3, 5, and 6 will be conducted at Caltech, together with a
subset of cross-lab replications.

RA. We request 25\% support for an RA. BSc or MSc level training in Psychology or
Behavioural Science will be sufficient. As outlined in Staff Duties, testing of participants one-
by-one, and testing 100 people is about 3 weeks of full time work. RA support will free
PDRA time for the more skilled writing and analysis.

Staff, Directly Allocated Posts

PI Stewart. We request 30\% support for Stewart. 15\%, half of this support, will be spent on
the laboratory programme, designing experiments and analysing data with the PDRA.
Stewart has developed a new statistical analysis for eye movement data (see two papers
forthcoming in Journal of Behavioural Decision Making) and this work will continue here.
The other 15\% is required for the ambitious impact work, which requires professorial level
and extensive liaison with the three current industry partners to negotiate access to data and
field trials, to set up access to three new partners, and to disseminate results back into
industry.

CI Chris Starmer. We will seek 10\% support for significant input from Experimental
Economist Prof. Starmer. Because of the interdisciplinary nature of the research, this
contribution is important for maximising the academic impact of the work beyond
Psychology.


Other Directly Incurred Costs

Computer resources. The computations required to estimate drift diffusion models and assess
model flexibility (e.g., with parameter space partitioning, Mullett & Stewart, 2015; Pitt,
Woojae, Navarro, & Myung, 2006) require serial search of model parameter spaces.
Compared to typical regressions which take fractions of a second, these methods take of the
order of days or weeks on, for example, a typical Intel 8-core i7-4770 CPU. We request
£2,000 for this dedicated computer machine.

Travel and Subsistence

We seek the costs of day visits to impact partners, all London based. We anticipate 6 visits per year x 2 people x 3 years. We will attend the SJDM conferences in 2016 and 2017. 2
people x 2 years (Vancouver and New Orleans) and the SPUDM conference (European city)
in 2017. These conferences are the most significant in the domain, giving coverage of
America and Europe. We seek the cost of a postdoc exchange with Prof. Colin Camerer at
Caltech. Caltech is the home of Camerer and Rangel labs, leaders in eye tracking economics.




From machine learned personality grant:

Justification of Resources
Staff, Directly Incurred Posts
Personality Psychology PDRA Skatova. We request 100% 36 months support for PDRA Skatova. A postdoctoral appointment is essential because the position requires high level of technical skills with psychometric approaches, as well as an advanced knowledge of the personality literature. The project is on the intersection of psychology and computer science, and thus experience across economics, computer science, and psychology that Skatova has will be essential. Further, experience translating output to industry, and experience of working with industry partners is essential. As outlined in Staff Duties, Skatova will design analyses with the investigators, and lead on the writing of publications. Skatova will implement the statistical analysis using the R programming language. Skatova will lead on contacts with industry partners (particularly Boots, Tesco and Barclays) and public policy collaborators (particularly the BIT and Which?). 
PDRA Computer Scientist. We request 100% 36 months support for a PDRA in the domains of data science, computer science, or machine learning. Postdoctoral experience of advanced algorithms and management of very large data sets is essential. The PDRA will lead on the writing of publications. The PDRA will supervise the management of data. The PDRA will develop the ML Personality Toolbox.
Both PDRAs will be involved in the Pathways to Impact demonstrators and workshops.
Lab manager. We request 10% 36 months support for a lab manager to support communication between the project team and project partners, implementing Data Management Plan as well as organization of Pathway to Impacts events including the workshops, MOOC and other public dissemination. 
Staff, Directly Allocated Posts
PI Stewart. We request 20% 36 months support for psychologist Stewart. 10%, half of this support, will be spent on the scientific programme, designing analyses of the data with the PDRA. Stewart has significant experience working with very large data sets, including financial transactions and police records. The other 10% is required for the ambitious impact work, which requires professorial level standing and extensive liaisons with the collaborators. Stewart will lead on contacts with the credit card industry and UBS. Stewart will oversee the second and third impact demonstrator. 
CI Goulding. We request 15% 36 months support for Goulding. 10% of this support will be spent integrating personality concepts with the development of machine learning algorithms for very sparse transaction data. The other 5% support is required for the impact work. Goulding will oversee the first impact demonstrator.
CI Chater. We request 10% 36 months support for Chater. Chater has extensive experience in mathematical psychology and the application to broad, domain-general principles of cognition. Chater also has significant experience in applying behavioural science to consumer behaviour as a director of Decision Technology. Finally Chater has led the “Mind is Flat” MOOC, one of Europe’s largest online courses, now in its fifth year and will supervise setting the MOOC for this project (see Pathways to Impact). 
Other Directly Incurred Costs
Computer resources. We request support for (University non-standard) PCs. These require encrypted 6 TB HDDs to store the retail data, and enough RAM (64 GB) to run analyses without swapping to the HDD. The OS will be hosted on modest SSDs. Only typical desktop CPUs are required (e.g., Intel 8-core i7-4770), with only standard graphics. A non-standard PC is required for Psychology PDRA at £1,600; and a specialist PC is required for CS PDRA at £2,484 with a larger SSD and a second processor to deliver enough threads for the machine learning algorithms. Since the project requires coordinating collaborators and project partners in various places across the country and well as geographically distributed Pathways to Impact activities, we request a laptops for Warwick Psychology PDRA who will be in charge with coordinating all partner activities to fulfil project plan, at £1,400. 
Cost of MOOC. We seek resources to develop a “Personality Data Science” MOOC: £10,000.  
Impact Workshops. We request support for four full one-day workshops, 20 people each at Shard, London.  
Travel and Subsistence. We seek the costs of travel between Warwick and Nottingham (75 trips over 5 investigators and 36 months) for project meetings between UoN and UoW investigators, as well as project partners who are based in Nottingham (Boots, Nutracheck), travel between Warwick and London to meet with project partners and for workshops (55 trips over 3 investigators and 36 months), travel between Nottingham and London to meet with project partners and for the workshops (48 trips, 2 investigators and 4 project partners from Nottingham, over 36 months). We also request cost for longer stay of Warwick PDRA in Nottingham to develop ML personality approaches with UoN CI and PDRA, and vice versa: travel and subsistence for 8 trips 2 weeks each Warwick to Nottingham for Warwick PDRA, and travel and subsistence for 8 trips 2 weeks each Nottingham to Warwick for Nottingham PDRA. 
Conferences. 4 CS conferences (2 UK, 2 europe, 2 overseas); 5 Psychology and Behavioural Science conferences (1 UK, 2 Europe, 3 overseas), 7 business science conference (2 UK, 3 Europe, 3 overseas) between 5 investigators. 



