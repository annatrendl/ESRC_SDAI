This project will use existing datasets from West Midlands Police (WMP) and the Crime Survey for England and Wales (CSEW) to investigate a number of aspects of domestic abuse.

Despite the extent and severity of resultant suffering, there is relatively little academic research focusing on domestic abuse in the context of the UK, perhaps due to the lack of reliable data sources. Using police data to understand domestic abuse is problematic due to the vast levels of underreporting, while insights from qualitative approaches are hard to generalise. In the UK, the most reliable source of data on domestic abuse is the Crime Survey for England and Wales (CSEW), an annual, cross-sectional representative survey collecting information on socio-economic characteristics as well as experiences of domestic abuse.  While the Office for National Statistics (ONS) regularly publishes findings from the CSEW, these are mostly exploratory and descriptive in nature. Despite the rich information provided by this survey, only a limited number of studies have utilised the CSEW to explore the predictors of domestic abuse victimisation in England and Wales. 

Our project fills this gap in our current understanding of domestic abuse by taking advantage of the wealth of information contained in the CSEW, the breadth and longevity of the WMP data, and supplementary metrics provided by a large UK bank. Applying sophisticated statistical techniques tailored to each analysis (including regression, random forest, and propensity score matching), allows us to provide insights in four main areas:

1) Exploring the characteristics of survivors and the predictors of serious harm: We will improve our understanding of risk factors for victims, and victims likely to be in most serious danger.

2) Understanding the decision to report and police mis-recording of domestic abuse: We will improve our understanding of which events and properties of the victim’s profile/environment encourage the decision to report ongoing abuse. In addition, we will improve our understanding of the types of victims or cases which are at risk of being misidentified, or not given appropriate resources after a report has been made.

3) The long-lasting effects of domestic abuse: We will quantify the long term effects on children who witness domestic abuse. For example, what effect does witnessing abuse have on the children's long term outcomes, and propensity to commit crime as adults.

4) Environmental factors influencing the prevalence of domestic abuse: What properties of the local environment and community predict the prevalence for different forms of domestic abuse. Here we will make use of data from a large UK bank to build a geographical measure for the levels of spending on gambling and alcohol in a given area (LSOA). We will also use this to build a measure for financial distress, specifically looking at the effect of changes in benefits such as the roll out of Universal Credit.

