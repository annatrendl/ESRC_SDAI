\documentclass[12pt, letterpaper]{article}
\usepackage{graphicx}
\usepackage{xcolor}
\usepackage[center]{caption}
\usepackage{hyperref}

\usepackage{apacite} 

\definecolor{green}{rgb}{0,1,0}
\newcommand{\NS}[1] {{\textcolor{green}{#1}}}
\newcommand{\TM}[1] {{\textcolor{orange}{#1}}}
\newcommand{\AT}[1] {{\textcolor{blue}{#1}}}


\begin{document}
\title{ESRC SDAI Application}
\maketitle

Domestic abuse is a multifaceted phenomenon that affects people from all walks of life, and is increasingly recognised as a major public policy concern across the world, including the UK. While anyone can become a victim of domestic abuse, women are disproportionately affected, with more than 25\% of women, and 15\% of men have reported to have experienced some form of domestic abuse since the age of 16 in England and Wales \cite{ONS}. The current legal definition of domestic abuse in the UK \cite{govuk} aims to capture its complex nature by recognising it as a pattern of behaviour encompassing a wide range of behaviours, including emotional, sexual, and physical abuse; threatening and intimidating behaviour, and coercive and controlling behaviour, which was introduced in 2015. 

Mental health implications? The most extreme cases of domestic abuse domestic homicides resulting in the loss of 70 lives in England and Wales in the period between April, 2017 and March, 2018, of whom 90\% of were women \cite{homic}.


Taken these together, the estimated economic cost of domestic abuse in England and Wales between April, 2016 and March, 2017 was £66 billion \cite{costs}. 



Recognising the severity of this widespread societal problem has resulted in the UK government launching several awareness campaigns to both encourage reporting and 




\bibliographystyle{apacite}
\bibliography{domesticabuse_refs}

\end{document}
