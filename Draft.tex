\documentclass[12pt, letterpaper]{article}
\usepackage{graphicx}
\usepackage{xcolor}
\usepackage[center]{caption}
\usepackage{hyperref}

\usepackage{apacite} 

\definecolor{green}{rgb}{0,1,0}
\newcommand{\NS}[1] {{\textcolor{green}{#1}}}
\newcommand{\TM}[1] {{\textcolor{orange}{#1}}}
\newcommand{\AT}[1] {{\textcolor{blue}{#1}}}


\begin{document}
\title{ESRC SDAI Application}
\date{}
\maketitle

\section{Intro}

Domestic abuse is a complex phenomenon affecting people from all walks of life. It is increasingly recognised as a major public policy concern in many countries, including the UK \cite{ep}. While anyone can become a victim of domestic abuse, women are disproportionately affected, with more than 25\% of women, and 15\% of men in England and Wales reported to have experienced some form of domestic abuse since the age of 16 \cite{ONS}. The current legal definition in the UK \cite{govuk} aims to capture the multifaceted nature of domestic abuse, by recognising that domestic abuse encompasses a wide range of behaviours, including emotional, sexual, and physical abuse, threatening, intimidating, coercive and controlling behaviour. 

Domestic abuse has substantial mental health implications, with an estimated three-quarter of survivors experiencing posttraumatic stress disorder symptoms, along with a significantly higher likelihood of reporting symptoms of anxiety and depression, compared to the general population \cite{ferrari}. In addition, witnessing domestic abuse at home can have severe developmental impacts on children, including an increased risk of experiencing mental and physical health problems and encountering difficulties in interpersonal relationships in later life, worse educational attainment, and increased likelihood of engaging in criminal behaviours \cite{callaghan}.

In the most extreme cases, domestic abuse can culminate in domestic homicide. In the period between April, 2017 and March, 2018, 70 people in England and Wales were killed by their current or former partner, 90\% of the victims were women \cite{homic}. While the pervasive problem of underreporting poses a significant obstacle to deriving reliable estimates of the true extent of the problem, the economic cost of domestic abuse in England and Wales between April, 2016 and March, 2017 was estimated to be £66 billion \cite{costs}. The largest component of this cost is represented by the physical and emotional consequences of abuse, reflected in a reduced expected quality of life for survivors. In addition, lost economic output resulting from missed workdays and reduced productivity, and costs to the health care system also significantly contribute to the overall figure. 


Recognising the severity of this widespread societal problem led the UK government to develop a strategic plan to tackle domestic abuse with a funding of £100m for the period between 2016 and 2020. The action plan focuses on a range of areas to reduce the prevalence of domestic abuse, including increasing awareness and willingness to report through education and information campaigns, introducing legal measures to increase victim safety (e.g., Domestic Violence Protection Order, Domestic Violence Disclosure Scheme, intervention programmes), and improve the responses of support services and health care professionals \cite{vawag}. \textit{Sentence about that while violent crime is decreasing, if repeat victims of domestic abuse is taken into account, domestic violence against women shows an increasing trend}

\section{Data}

Our datasets include the 




\section{Aims}

\subsection{Victim-offender profiles - comparison of CSEW and WMPD}

We first explore the characteristics of offenders and victims, focusing on gender, age and ethnicity. 


Aim 1: Who perpetrates? Who becomes a victim? - our police dataset allows us to explore what other types of criminal behaviours abusers are likely to engage in/be a victim of; same for victims with police and CSEW; victim/perpetrator characteristics (gender/ethnicity); deprivation and domestic abuse? Find out what kind of cases get reported. Break down the 17\% based on socioeconomic and incident characteristics.


\subsection{Reporting behaviour}

Increasing victim's willingness to report domestic abuse is key in reducing prevalence and preventing escalation into domestic homicides. We will use the CSEW to understand the demographic and socioeconomic predictors of the decision to report the abuse to the police and identify groups who are the least likely to report. We can contrast and validate these findings using our crime dataset.


Our unique crime dataset will further allow us to follow victim-offender pairs over time and explore how the severity and nature of reported abuse changes over time. We will also be able to identify high-risk days of the year (e.g., birthdays, Christmas, Halloween) by the characteristics of the victim-offender pair and the pattern and duration of abuse. Insights from this investigation can inform decisions about the optimal timing and target audience for domestic abuse awareness campaigns. 

Separation is one of the biggest predictors of abuse resulting in serious injury. Subject to whether we can get data on the DVPOs (Domestic Violence Protection Order) and DVPNs (Domestic Violence Protection Notice) issued in the West Midlands in the relevant period, we could explore the offender characteristics that predict the likelihood of sustained perpetration after separation, and assess the overall efficacy of these new preventive measures.



\subsection{Association with other criminal behaviours}


Aim 3: explore the lasting effect of domestic abuse on the victims and their children (do we see kids having behavioural problems? No control group unfortunately)- CSEW question about gang membership - are those who report having a child gang member more likely to be victims of domestic abuse? Following people over time (Dynamic topic modelling for victims)- what’s the temporal order? 

\subsection{Protective factors - shouldn't this go into the victim-offender profiles?}
Aim 4: Protective factors and interventions. various aspects of alcohol-involvement. Is it worse when alcohol is involved? {Outlawed: Coventry domestic abuse interventions?} Protective factors identified in CSEW data---e.g., having a job, car, no litter outside, living with other relatives,



\bibliographystyle{apacite}
\bibliography{domesticabuse_refs}

\end{document}
