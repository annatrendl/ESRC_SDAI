\documentclass[11pt, a4paper]{article}

% Helvetica
\usepackage[scaled]{helvet}
\renewcommand\familydefault{\sfdefault} 
\usepackage[T1]{fontenc}

% 2cm margins
\usepackage[margin=2cm]{geometry}

\usepackage{graphicx}
\usepackage{xcolor}
\usepackage[center]{caption}
\usepackage{hyperref}

\usepackage{apacite} 
%\usepackage{setspace}
%\doublespacing

\definecolor{green}{rgb}{0,1,0}
\newcommand{\NS}[1] {{\textcolor{green}{#1}}}
\newcommand{\TM}[1] {{\textcolor{orange}{#1}}}
\newcommand{\AT}[1] {{\textcolor{blue}{#1}}}


\begin{document}
\title{ESRC SDAI Application}
\date{}
\maketitle

\section{Intro}

Domestic abuse is a complex phenomenon affecting people from all walks of life. It is increasingly recognised as a major public policy concern in many countries, including the UK \cite{ep}. While anyone can become a victim of domestic abuse, women are disproportionately affected, with more than 25\% of women, and 15\% of men in England and Wales reported to have experienced some form of domestic abuse since the age of 16 \cite{ONS}.  The current legal definition in the UK \cite{govuk} aims to capture the multifaceted nature of domestic abuse, by recognising that domestic abuse encompasses a wide range of behaviours, including emotional, sexual, and physical abuse, threatening, intimidating, coercive and controlling behaviour. 

Domestic abuse has substantial mental health implications, with an estimated three-quarter of survivors experiencing posttraumatic stress disorder symptoms, and a significantly higher likelihood of reporting symptoms of anxiety and depression, compared to the general population \cite{ferrari}. The long lasting impacts of domestic abuse are not limited to the direct target of abuse. Witnessing domestic abuse at home can have severe developmental impacts on children, including an increased risk of experiencing mental and physical health problems and encountering difficulties in interpersonal relationships in later life, worse educational attainment, and increased likelihood of engaging in criminal behaviours \cite{callaghan}.

As the legal definition reflects, one characteristic of domestic abuse that differentiates it from other violent crimes is its repeated nature. Estimates show that on average, victims live in the abusive relationship for 2.7 years, experiencing an estimated 50 cases of abuse before getting effective help \cite{SafeLives2015}. The most reliable statistics on domestic abuse in the UK is the Crime Survey for England and Wales (CSEW; \citeNP{ONS}), a victimisation survey including a self-completion module on domestic abuse. According to the CSEW, of those respondents who experienced domestic abuse between April 2017 and March 2018, only 17\% reported it to the police \cite{ONS}. This extremely high level of underreporting is another characteristic that is specific to domestic abuse. 


In the most extreme cases, domestic abuse can culminate in domestic homicide. In the period between April, 2017 and March, 2018, 70 people in England and Wales were killed by their current or former partner, 90\% of these victims were women \cite{homic}, demonstrating that domestic abuse is a fundamentally gendered phenomenon. While the pervasive problem of underreporting poses a significant obstacle to deriving reliable estimates of the true extent of the problem, the economic cost of domestic abuse in England and Wales between April, 2016 and March, 2017 was estimated to be \pounds 66 billion \cite{costs}. The largest component of this cost is represented by the physical and emotional consequences of abuse, reflected in a reduced expected quality of life for survivors. In addition, lost economic output resulting from missed workdays and reduced productivity, as well as costs to the health care system also significantly contribute to the overall figure. 


Recognising the severity of this widespread societal problem led the UK government to develop a strategic plan to tackle domestic abuse with a funding of £100m for the period between 2016 and 2020. The action plan focuses on a range of areas to reduce the prevalence of domestic abuse, including increasing awareness and willingness to report through education and information campaigns, introducing legal measures to increase victim safety (e.g., Domestic Violence Protection Order, Domestic Violence Disclosure Scheme, intervention programmes), and improve the responses of support services and health care professionals \cite{vawag}. 

%These interventions are crucial in light of the fact that after a long decline, domestic violence against women has significantly increased between the period of 2008 and 2014 \textit{ref}. 

\newpage


\section{Data}


The first dataset is provided by the West Midlands Police, and includes all recorded crimes and incidents from the period between January, 2010 and October, 2018. The West Midlands Police is the third largest police force in England, serving a population of 2.9 million \textit{(ref)}. Crimes that have a domestic abuse marker indicate cases of domestic abuse that meet the criteria for notifiable offences in the UK, whereas domestic abuse incidents refer to cases that do not qualify as a crime. About 31\% of all crimes and incidents have a domestic abuse marker. For each record in this dataset, we have information about the exact location and time of the incident or crime, the gender, age and ethnicity of the offender and victim, and the severity of the injury sustained by the victim, if any. The first and last occurrence of the offence, as well as the exact time of reporting is also recorded. Each person in this dataset has a unique person identifier, allowing us to follow people over time. \textit{Sentence about access to this dataset.}

The second dataset is the Crime Survey for England and Wales (CSEW), which is an annual, cross-sectional representative survey collecting information on victims of crime across England and Wales, including 30,000--40,000 households every year \textit{(ref)}. The survey has a self-completion module containing questions about the respondent's experiences of domestic abuse throughout their life (since the age of 16), and in the past 12 months in particular \textit{(ref)}. Given the serious problem of underreporting uniquely characteristic to domestic abuse, the CSEW is the most reliable source of information on the prevalence of domestic abuse in England and Wales.   since when?

The two datasets have different merits, and combining them allows us to gain a deeper understanding the characteristics and dynamics of domestic abuse, and deliver policy-relevant insights. The CSEW is more likely to give a better estimate of the true prevalence of domestic abuse, and contains rich demographic, socio-economic and geographic (LSOA level) information on the respondent (victim). However, owing to the fact that it is a victimisation survey, it does not contain any information on the perpetrators and has limited longitudinal information. In contrast, the crime dataset has information (age, gender, ethnicity) on both the victim and the offender, and records all reported incidents for the same victim-offender pair, allowing us to conduct a quantitative analysis of the dynamics of abusive relationships and identify the predictors of escalation. Information on the exact timing of the abuse and the time of reporting will help us identify high-risk times of the year and understand the factors influencing the decision to report. Information on alcohol involvement is more likely to be accurate in the crime dataset than in the self-completion survey. Furthermore, the crime dataset provides us with information on other criminal behaviours of the offender and victim.  We do not intend to link the two datasets.

\newpage

\section*{Research Programme}

In outlining the research program we specify X core regression analyses, which we will preregister ahead of beigging research on the OSF. These will be complemented with the robustness checks and further exploratory regressions (see, Trendl Football, for an example of this approach). 

\subsection*{1. Understanding domestic abuse: victim characteristics, extent of domestic abuse and police mis-recording}

 \textcolor{red}{\textit{Short intro about aims, highlighting added value.}}
In our first research project, we will be investigating three main research questions. First, we will use the CSEW dataset to identify the demographic and socio-economic predictors of domestic abuse victimisation, and validate the CSEW estimates using the WMP data. Second, we will estimate the true extent of domestic abuse victimisation. Given the multifaceted nature of domestic abuse, and the pervasive problem of underreporting, it is very hard to reliably estimate the extent of domestic abuse victimisation and the subsequent societal costs. We wish to do so by combining the strengths of the two datasets (that is, the CSEW provides a more reliable measure of the true rate of victimisation across different offence types, while the WMP provides unique information on whether various types of offences were domestic abuse-related). Exploring the risk factors in domestic abuse victimisation, and the true extent of the problem of underreporting is crucial for designing effective interventions to reduce the risk of escalation that may culminate in domestic homicide. Finally, we will estimate the extent of police mis-recording of domestic abuse. Increasing victim's trust in the police through improving law enforcement response to reported incidents is key in encouraging victims to report domestic abuse and prevent further harm. By contrasting estimated levels from reported incidents within the West Midlands from the two datasets, we can identify any reporting gaps between the two sources of data that might arise from police mis-recording.

\textcolor{red}{\textit{What we know so far}} 

Aim 1. \href{https://www.ons.gov.uk/peoplepopulationandcommunity/crimeandjustice/articles/domesticabusefindingsfromthecrimesurveyforenglandandwales/yearendingmarch2018}{CSEW} : 7.9\% women and 4.2\% men reported to have experienced domestic abuse in the past year; women between 20-24 are most likely to have experienced domestic abuse; for men, age doesn't represent a significant risk factor;  divorced or separated status are significant risk factors across both sexes; long-term illness or disability is a significant predictor across both sexes; single parent households are the most vulnerable; women in the lowest household income bracket are the most likely to have experienced domestic abuse, the same is true for men but the pattern is less pronounced; only 17\% of people who have experienced domestic abuse in the past year have reported to the police, there is no significant difference in this proportion across genders.   
single mothers and mothers with young children are more at risk

Aim 2. The prevalence of DA is usually based on the CSEW, so I think the second analysis can add value.

Aim 3. \href{https://www.justiceinspectorates.gov.uk/hmicfrs/news/news-feed/little-evidence-of-improvements-in-crime-recording-at-west-midlands-police/}{HMICFRS about West Midlands mis-recording}

\textcolor{red}{\textit{Detailed plan}}

\textbf{Aim 1.} First, using the CSEW data, we will conduct a series of statistical analyses to provide a deeper understanding of the demographic and socio-economic factors predicting domestic abuse victimisation. Using a logistic regression approach, we will explore the factors predicting individual-level domestic abuse victimisation, depending on the type of abuse ever suffered (physical abuse, threats, sexual abuse). The CSEW provides a wealth of information on the characteristics of the respondent, including age, sex, marital status, number of children, ethnicity, education, employment, income, benefit history, physical and mental health, frequency of going out, house and car ownership, self-reported well-being, and frequency of drug and alcohol use. Using this wide range of individual-level information, we will evaluate the importance of each of these variables in predicting the likelihood of domestic abuse victimisation. \textit{Could we also use a e.g. a random forest classification algorithm and look at variable importance? this way we could also look at interactions? does clustering make sense here to explore structure?}  These analyses will represent the first extensive quantitative investigation of the individual-level predictors of domestic abuse victimisation in England, using a national-level, large representative sample like the CSEW. We are particularly interested in the association between experiences of domestic abuse, number of children and financial resilience, as previous research has suggested that access to resources is an important factor in escaping the cycle of abuse, especially for women with children.

Second, we can validate some of these findings using the WMP data, since the CSEW contains information on whether the victim reported the domestic abuse to the police. The WMP contains information about the sex and age of victims. We can contrast the sex-age victimisation numbers from the WMP with corresponding population estimates to validate the findings of the CSEW \textit{(we were initially talking about using victims of burglary, but maybe this is better?)}. We can also complement this analysis with data on the offender's sex  and age, to estimate the prevalence of different forms of reported domestic abuse (child to parent violence, parent to child violence, intimate partner violence) and contrast it with estimates based on self-reported data from the CSEW.

%As mentioned, domestic abuse is a complex phenomenon that encompasses a number of notifiable criminal offences that are usually repeat incidents in nature (e.g., violence against the person including physical or verbal abuse, sexual offences, criminal damage, stalking and harassment). However, the CSEW mainly focuses on 

\textbf{Aim 2.}  While a unique strength of the CSEW is that it provides reliable estimates of the proportion of the population who have been a victim of a certain crime, it cannot capture the true extent of domestic abuse victimisation for several reasons. First, because it does not collect detailed information on repeat domestic abuse victimisation (an inherent characteristic of domestic abuse cases), but only on the number of incidents involving physical violence perpetrated by household members. In addition, victims who suffer frequent, ongoing abuse are less likely to accurately remember the exact number of incidents. Second, because it only focuses on certain sub-categories of domestic abuse (physical violence, sexual assault, stalking), while ignores others (criminal damage, verbal abuse, financial abuse, burglary). In addition, it does not collect information on whether different offences against the same victim were perpetrated by the same offender. In contrast, the WMP data allows us to tell if a reported case was domestic abuse related, regardless of the actual offence committed. In addition, it allows us to identify various incidents where the same victim-offender pair were involved. We can estimate the co-morbidity of self-reported victimisation by crime type using the CSEW. We can contrast this with the estimated co-morbidity of reported victimisation from the police data. \textit{is this silly? then we can maybe estimate what percentage of true victimisation is domestic-abuse related? the £66 billion cost estimate is based on the CSEW}


%It is common that individuals are more likely to be a victim of crime if they have previously been a victim of crime, and if they have perpetrated crimes. It is also more likely that an individual will commit criminal activities if they have done so before, and if they have previously been the victim of crimes. We will use our datasets to improve our understanding of the temporal patterns of victims and perpetrators, particularly around domestic abuse. The high level of detail in the CSEW allows us to examine the comorbidity of domestic abuse with other crimes during the previous year, and since the age of 16. 



\textbf{Aim 3.} Police forces in England, including the West Midlands Police, have previously been criticised for mis-recording violent cases of violent crime, including \href{https://www.bbc.co.uk/news/uk-england-468676578}{domestic abuse.} Mis-recording these already heavily underreported crimes decrease victim's trust in the Criminal Justice System and their willingness to report subsequent incidents. By contrasting the extrapolated estimates from the CSEW (which is a representative sample) with WMP data on the true number of reported cases, we can estimate the true extent of the mis-recording of domestic abuse cases by the WMP. Given the repeated cross-sectional nature of the CSEW, and the longitudinal nature of the WMP, we can quantify the proportion of mis-recorded incidents over time. We can complement this investigation with analysing the self-reported outcome of incidents that have been reported to the police from the CSEW.



%co-occurrence of other types of crimes - we can contrast results from the CSEW with results from our crime data - e.g., if we see that those reporting domestic abuse have also experienced criminal damage in the CSEW, we can check in the WMP data if criminal damage or home theft is usually a spillover effect from domestic abuse ; deliberate violence; account theft



%We will use the CSEW to understand the demographic and socioeconomic predictors of the decision to report the abuse to the police, and identify specific groups who are less likely to report. We are able to do this because CSEW respondents indicated whether they reported the abuse to the police  and whether the police came to know of the abuse in another way (Question X). 
%
%Regression 1: data --- subset of people who in CSEW responding that they have been abused
%
%	abuse or not ~
%   	[demographics] age + sex + marital status + earnings + ...
%	other crime dummies + ...
%
%
%	unreported abuse / reported abuse ~ 
%	[demographics] age + sex + marital status + earnings + ... 
%	other crime dummies + ...
%	[information about abuse] level of harm + ...
	

%Respondents also answered questions indicating the level of harm that they have suffered, which will allow us to estimate how reporting and police awareness depend upon the level of harm. \NS{Anna: broaden this paragraph to be about all of the covariates}

%Linear probability model (and logit model?)

%We can contrast and validate these findings using the WMP dataset. 
%The WMP data are cases that have been reported to the police. The CSEW data allow an estimate of the true underlying prevalence of domestic abuse. The discrepancy between the two datasets allows us to estimate the reporting gap. 
%That is, we can use the WMP dataset to provide a second estimate of the degree of underreporting, indpendent of the CSEW respondent's indications of report. Discrepancy between reporting prevelence estimated from CSEW and the number of cases in the WMP data give important information about the police's recording and categorisation of domestic abuse cases---do police flag all of the cases of domestic abuse?  https://www.bbc.co.uk/news/uk-england-46867657

%While the CSEW provides us with accurate information on the characteristics of victims, it does not contain information on the perpetrators. Combining the two datasets helps us find commonalities between the perpetrators of abuse cases that are.


%Domestic abuse doesn't discriminate---well we can estimate this by comparing the 7.7\% for females and 4.4\% for males in the past year with the size of the coefficients.

%We can describe who is abused and their demographics in CSEW, CSEW who report, and in WMP.

%We can also compare those abused in WMP data with any other reference group in the WMP data. (e.g., burgled) and then use DA-burgled in CSEW vs WMP to back out stuff.


\textcolor{red}{\textit{Core outcomes}} 
A deeper understanding of the socio-economic predictors of domestic abuse victimisation. The first estimation of the true extent of domestic abuse-related victimisation across all offence types. An estimation of the extent of police (WMP) mis-recording of domestic abuse over time. 

\newpage


\subsection*{2. Predicting serious harm and understanding the decision to report}


 \textcolor{red}{\textit{Short intro about aims, highlighting added value.}}


\textcolor{red}{\textit{What we know so far}}

\href{https://www.thehotline.org/2018/09/28/escalation/}{Escalation} escalation often happens when the abuser feels like they are losing control (e.g., because the victim threatens to leave); separation is an important risk factors of serious harm;



\href{http://www.safelives.org.uk/policy-evidence/about-domestic-abuse/how-long-do-people-live-domestic-abuse-and-when-do-they-get}{Reporting} 


\textcolor{red}{\textit{Detailed plan}}

does the level of harm depend on the length of time spent in the relationship, separation, 

Understanding what affects the decision to report is particularly important since underreporting represents the biggest obstacle to effectively tackling domestic abuse.

\textbf{Aim 1.} First, we will use the CSEW, and a multinomial logistic approach to understand how the level of harm suffered (threats, minor or serious injury) depends on the individual-level characteristics of the victim (especially focusing on the issue of separation and divorce, as evidence suggests it is a significant risk factor predicting serious harm). We will also explore whether non domestic abuse-related victimisation is predictive of serious harm. We will know the level of harm suffered, because the CSEW asks whether the respondent needed to see a doctor, nurse or other health worker, or needed to take time off work because of the abuse. 

Second, using the WMP data, we can analyse the temporal patterns of reported domestic abuse incidents by following  23\% of the 182,646 victim-offender pairs over an eight-year-long period. 

We can predict incidents with serious harm, using the type of offence occurred previously, criminal history of the offender. number of days elapsed. This will allow us to identify and commonalities in the pattern of domestic abuse cases that lead to escalation in the form of serious harm. We can also identify and characterise high-risk perpetrators who breach court orders designed to protect the victims from further harms.

%We can examine whether a recorded victim of domestic abuse was previously recorded as a victim in other crimes, and if they were subsequently a victim of other criminal behaviours. Identifying risk factors is crucial for police services who want to identify individuals at risk of being future victims, so that they can target interventions and give the best advice possible to reduce an individual's risk. In addition, we can measure the relationship between experiencing domestic abuse, and the likelihood of being recorded as a suspect in other crimes. It is possible that some criminal behaviour is a precursor to experiencing domestic abuse whilst others follow it. 



\textbf{Aim 2.} First, we will use the CSEW, and a logistic approach to understand how the decision to report to the police (or seeking any external help) depends on the individual-level characteristics of the victim (especially focusing on mental and physical health, financial independence and whether the victim has children) and the abuse suffered. We can complement this analysis with the WMP data, 

children's safety as a concern

Second, using the WMP data, we will also be able to identify high-risk days of the year (e.g., birthdays, Christmas, Halloween) by the characteristics of the victim-offender pair and the pattern and duration of abuse. Insights from these analyses can inform decisions about the optimal timing and target audience for domestic abuse awareness campaigns. 


\textcolor{red}{\textit{Core outcomes}}

%Aim:

%The CSEW respondents indicated the level of harm, and we can use the CSEW to estimate the cross-sectional association between the most serious harm and other comorbid factors like alcohol, other crimes, 

%Regression 3: type of abuse ~ 
%	visit hospital ~ 
%	mental health ~ 

%Separation is one of the biggest predictors of abuse resulting in serious injury. \textit{Subject to whether we can get data on the DVPOs (Domestic Violence Protection Order) and DVPNs (Domestic Violence Protection Notice) issued in the West Midlands in the relevant period, we could explore the offender characteristics that predict the likelihood of sustained perpetration after separation, and assess the overall efficacy of these new preventive measures. If not, we can still use info on location + breaches of non-molestation orders?}

%But cross-sectional data present a particular problem for making casual claims: For example, incidents involving drugs or alcohol intoxication: some individuals may end up in an abusive relationship because of an existing dependency, or a dependency may develop as a counterproductive strategy of coping with the abusive relationship. 



%Should you be identifying the couples most at risk of serious harm from time-invariant properties, or should you be following couples over time and using time-varying properties to spot when to intervene? 

%The WMP dataset will further allow us to follow victim-offender pairs over time, and explore how the severity and nature of reported abuse changes over time. 
%In the WMP data we can follow X victim-perp pairs for X years, with each pair appearing X times on average in the data.

%Regression 4: offense classification[proxy for harm] ~ .
%	type of abuse ~ .
%
%Future victim behaviours. Future perpetrator behaviours
%
%serious event at t ~ stuff at t-1  + perpetrator previous crimes
% and 	a model with pair fixed effects
%
%Murders are very rare, so we are adopting a ``near miss'' airline / nuclear industry strategy of predicting the more prevalent but less serious incidents. 
%
%Say pair fixed effects control for all non-time varying characteristics. With and without pair fixed effects let's use see how much state vs trait matters for prediction. 


\newpage

\subsection*{3. The long-lasting effects of domestic abuse}

it's impossible to establish causality

 \textcolor{red}{\textit{Short intro about aims, highlighting added value.}}
Domestic abuse has long-lasting adverse effects on victims and those close to them. In this research project, our is aim to explore some of these consequences on the direct victims of domestic abuse and the children who live in the same household. Gaining a deeper understanding of the tangible, far-reaching consequences of domestic abuse will help to design victim support programmes and quantify the societal harm caused by it.


\textcolor{red}{\textit{What we know so far}}
\href{https://www.womensaid.org.uk/information-support/what-is-domestic-abuse/the-nature-and-impact-of-domestic-abuse/}{Women's aid, effect on victims} increased likelihood of mental health problems (isolation, low self-esteem), alcohol and drug dependency, financial consequences due to financial abuse or losing a job as a result of the abuse, fighting back?

\href{https://uksaysnomore.org/effects-of-domestic-abuse-on-children/}{Effect on children I} witnessing violence is major predictor for PTSD and depressive symptoms in children; behavioural problems, bullying (either victims or perpetrators)

\href{https://www.psychologytoday.com/gb/blog/progress-notes/201902/alarming-effects-childrens-exposure-domestic-violence}{Effect on children II} cognitive development is affected, learning difficulties are more common, difficulties in social relationships, withdrawal, antisocial behaviours, they are more likely to become both perpetrators and victims of domestic abuse; protective factors: literacy, intelligence, social competence, supportive relationship with at least one influential adult 


\textcolor{red}{\textit{Detailed plan}}

\textbf{Aim 1.} First, using the CSEW, we will explore the long-lasting effects of experiencing domestic abuse. In the CSEW, in two separate questions, respondents are asked whether they have experienced any form of domestic abuse (use of violence, sexual abuse, threatening behaviour, stalking) since the age of 16 or in the last 12 months. This way we can identify respondents who have suffered specific forms of abuse in the past, but not in the last 12 months. Using propensity score matching, we can then compare their physical and mental health-related outcomes, and alcohol and drug consumption with those, who have never experienced any form of abuse, but otherwise have similar socio-economic characteristics. This will allow us to estimate the effect of past abuse on present outcomes whilst controlling for individual- and macro-level characteristics.  

\textbf{Aim 2.} Using data from the 10-15 year old questionnaire of the CSEW, which can be linked to the adult questionnaire, we can examine the effect of domestic abuse on children living in such households. The questionnaire records information on whether the child had been physically or verbally abused by someone (and if that person lives in the household), as well as their experiences with bullying, carrying knives and information on gang membership, school truancy and health. We can explore the effect of domestic and child abuse on these outcomes through propensity score matching.

Domestic abuse often co-occurs with child abuse. We can use the WMP data to estimate this co-morbidity, and identify under 18s who had been a victim of a domestic abuse offender. No control group. 



%Using data from the CSEW and well as population data of the West Midlands, we can estimate how many children and young people below 18 in each LSOA of the West Midlands are at risk of witnessing domestic abuse at home each year. 

%\textit{The problem is that we don't know the address of the young offenders. Could we the location info on where they committed the offences as a proxy of the LSOA where they live? not sure if this is viable; defining a gang member is problematic in the crime data - if something serious is committed probably only one of them will be charged with it, maybe we should just concentrate on criminal activities; I was trying to identify children who were victims of something happening at an address where DA happened and perpetrated by a DA offender - it's very narrow and excludes children who *just* witnessed abuse; however, the location id doesn't seem to specific to one place - one id can correspond to a road, dwelling, shop etc.}


%We can use school exclusion data - unfortunately this data is only available on the local authority and school-type (state-funded primary, secondary, special) level...

%There's a children's questionnaire, and they ask about gang membership. However, the sample size is very low (3,100 for the whole country). 

%of the 119,539 under 18s who were victims of crimes (domestic/child abuse, or other more serious physical/sexual violence) in the dataset, 11\% were later offenders with an average of 2 offences (mostly domestic abuse, physical violence, drug possession and theft) - these are only the kids we have evidence were traumatised at some point

%of the 149,759 da offenders, 19\% have an incident against an under 18 





%Aim:
%
%CSEW regression: child gang membership ~ any domestic abuse or reported DA [2x2 table of abuse/no abuse by gang / no gang]
%Also estimate, given gang membership what fraction are DA---which links ot WMP analysis below
%
%It also contains information about  multiple members of the household. This allows us to examine whether young individuals living in a household with a victim of domestic abuse are more likely to be victims or perpetrators of other types of crime. By using other questions in the CSEW, we can also measure whether these children are more likely to perpetrate crimes, and to be involved in gang related activities.
% 
% explore the lasting effect of domestic abuse on the victims and their children (do we see kids having behavioural problems? No control group unfortunately)- CSEW question about gang membership - are those who report having a child gang member more likely to be victims of domestic abuse? Following people over time (Dynamic topic modelling for victims)- what’s the temporal order? 
%
%Match on name, location, and date of birth in WMP data to see effect on children
%
%WMP regression:
%
%take every juvinile in a gang and try to match all to domestic abuse cases----estimates of what fraction of gang members have co-morbid domestic abuse
%
%What fraction of reported child abuse has comorbid DA; what fraction of DA has comorbid child abuse
%
%Using school census we can go futher and estimate frequencies in a 2x2 child abuse  by domestic abuse table---providing we can match children to mothers.

\textcolor{red}{\textit{Core outcomes}}

\newpage

\subsection*{Environmental factors}

\textcolor{red}{\textit{Short intro about aims, highlighting added value.}}
In this research project, we will be using the CSEW and WMP datasets, as well as external sources of data to investigate some of the environmental predictors of domestic abuse. First, using data from the CSEW and drawing on previous research on the topic, we will be looking at the association between neighbourhood-level characteristics, and the prevalence of domestic abuse as well the decision to report. We will complement this with an additional analysis on WMP and census data. We will be able to account for the dynamic nature of these neighbourhood characteristics, by using longitudinal and repeated cross-sectional datasets. Understanding how neighbourhood-level characteristics affect the prevalence of domestic abuse and reporting behaviour is crucial for the most effective distribution of resources to help victims and targeted awareness campaigns. 

Second, we will be looking at how time-varying, exogenous factors affect domestic abuse. We will first explore the effect of local events that increase alcohol-consumption, since our previous study have found a significant relationship between alcohol-related domestic abuse and football. Second, we will conduct two studies to understand the relationship between financial stress and the prevalence of domestic abuse. First, we will explore the association between gambling accessibility in the local area, and reported numbers of domestic abuse. Second, we will examine how the roll-out of the Universal Credit system affected levels of domestic abuse across the West Midlands. Apart from delivering important policy-relevant insights, these investigations will also aim to inform the local police about the optimal allocation of police resources. 



\textcolor{red}{\textit{What we know so far}} interestingly, there are no studies about this using UK data; neighbourhood-level predictors of domestic abuse: social disorganization theory (socio-economic disadvantage and residential instability disrupt social bonds a limit collective ability to maintain control and recognise IPV); cultural norms through social learning process (high IPV neighbourhoods where people observe influential others being rewarded for engaging in IPV, whereas in low IPV neighbourhoods perpetrators get socially ostracized); women's empowerment reduces IPV, especially financial independence;  abuser's lack of employment and alcohol consumption can be a risk factor; alcohol outlet density; churches, playgrounds, community networks might foster community cohesion and reduces IPV;  however, in deprived neighbourhoods, where IPV is accepted social interactions can increase IPV

\href{https://www.sciencedirect.com/science/article/abs/pii/S0277953616302891}{The relationship between electronic gaming machine accessibility and police-recorded domestic violence: A spatio-temporal analysis of 654 postcodes in Victoria, Australia, 2005–2014}

\href{https://journals.sagepub.com/doi/abs/10.1177/1525107115623938}{Economic Stress and Domestic Violence: Examining the Impact of Mortgage Foreclosures on Incidents Reported to the Police }

\href{https://journals.sagepub.com/doi/abs/10.1177/1524838012445641?journalCode=tvaa}{
The Impact of Neighborhoods on Intimate Partner Violence and Victimization
}

the type sof social tie matters: friendship ties reduce the risk of IPV, family ties have no effect

\href{https://journals.sagepub.com/doi/abs/10.1177/1524838013515758}{Neighborhood Environment and Intimate Partner Violence: A Systematic Review }

\textcolor{red}{\textit{Detailed plan}}

\textbf{Aim 1.} First, we will use the CSEW to investigate if there are significant associations between the prevalence of domestic abuse, the willingness to report, and the environmental characteristics of the neighbourhood they live in. Previous literature has suggested various pathways of how neighbourhood characteristics affect domestic abuse, but the question has not been explored in a UK context yet. Our rich data allows us to determine the relative explanatory power of these. For example, social disorganization theory posits that neighbourhood disadvantage is associated with an increased risk of domestic abuse through reduced collective efficacy. Another potential pathway is social norms, suggesting that acceptance and normalisation of violence within a community can encourage domestic abuse. Using a logistic regression, we will explore how the interviewer's perception of the street (signs of rubbish, vandalism, and the general condition of houses, \textit{member of Neighbourhood Watch?}) as well as the respondent's connection and attitude towards the neighbourhood (length of time living in the local area, noisy neighbours, rubbish lying around, teenagers hanging around, vandalism, drunken and anti-social behaviour, drug trafficking, abandoned cars, speeding traffic, police presence in the area, worries about crime levels) predict domestic abuse victimisation and trust in CJS, reflected in the propensity to report abuse. Second, using the WMP and the census, we can compliment this with an LSOA-level spatial regression, where we use the socio-economic characteristics (average income, education levels and benefit dependency) of the area to predict the number of reported abuse cases. Using a spatial regression will account for the spatial dependency between our observations and improve the reliability of our estimates. \textit{Not sure whether we can do the same with the CSEW, probably not on the LSOA level, but maybe on the MSOA level?} This analysis, contrasted with that outlined in project 1 about the individual-level predictors of domestic abuse, will provide us with the most comprehensive understanding on the risk factors of domestic abuse within the context of the UK.

\textbf{Aim 2.} In this section, we will be looking at time-varying environmental factors that may influence domestic abuse. This part of the project will benefit from a dataset containing credit card and current account spending of x customers across the UK, to which we will have access as part of a data sharing agreement with a very large UK bank. \textit{I'm not sure about the specifics of this, would we combine this with the WMP or CSEW? I guess we won't know the card owner's address?} First, by using the WMP and credit card data, we will identify time-specific changes in alcohol consumption in parts of the West Midlands by identifying externals events that may affect it (e.g., local festivals, weather, bank holidays, sport tournaments), and investigate their effect on alcohol-related domestic abuse. Previously, we have found a 60\% increase in alcohol-related domestic abuse when the England national football team won, highlighting the profound effect exogenous events can have on the propensity for violence. 


  The recent reforms of the UK benefit system, in the form of the introduction of the Universal Credit (UC) had been criticised widely, due to the temporal financial strain it imposes on the least financially resilient people in society. Drawing on previous findings about the link between financial stress and domestic abuse, we will explore what effect the roll-out of the UC across the West Midlands had on the reported number of domestic abuse cases. The fact that the exact date of the roll-out varied across the seven metropolitan boroughs within the county allows for a more precise estimation of the effect of UC on the reported number of domestic abuse incidents in the West Midlands. For this estimation, we will use data on the proportion of benefits claimants in each LSOA. \textit{I know we were talking about using the credit card data for this one, but I am not sure what was the exact plan}. Gambling is increasingly recognised as a serious health concern across the UK, and can have an adverse impact on family finances. To assess the link between gambling accessibility and the prevalence of domestic abuse, we will use data on the number of licensed betting shops in the West Midlands within a spatial regression approach.

\href{https://secure.gamblingcommission.gov.uk/PublicRegister}{List of licensed gambling businesses} We can also contact the local authorities, as they should have an up-to-date list


\textcolor{red}{\textit{Core outcomes}} The first extensive analysis of the complex individual-- and neighbourhood level predictors of domestic abuse within the UK. An exploration of the exogenous, time-varying factors affecting levels of reported domestic abuse (mostly through alcohol). An examination of how gambling accessibility and temporal financial stress affect the reported number of domestic abuse cases in the West Midlands. 



%Unit of analysis is large geographical area in a year 
%Time span 2005-
%
%DA or not ~
%
%Given DA: report or not ~
%
%Fixed effects for region and year
%
%matched in socieoconomic
%
%
%
%WMP analysis
%
%Unit of analysis is LSOA (MSOA?) per month or per quarter
%Time span - 2010-2020
%
%Number of DA cases ~ 

%The CSEW allows us to identify the demographic and socio-economic characteristics of victims who are the least likely to report, and therefore should be in the focus of support services. We can validate and contrast these findings with insights from a spatial regression analysis of the crime data, using low-level geographical area information from the census. 

%Extend regression model X to include region fixed effects?
%
%
%\textit{A sentence about what earlier investigations have found about particular risk factors, including socio-economic deprivation.} 

%deprivation is a risk factor for women
%
%Australia study on domestic abuse and proximity to fixed-odds betting terminals, and the one on offlicences.
%
%anything that affects propensity to drink/financial stress
%
%neighbourhood characteristics (deprivation), fixed odds betting terminals, alcohol outlets, events (football, festivals)
%
%\href{https://www.birminghammail.co.uk/news/midlands-news/universal-credit-rolled-out-across-14142901}{Universal credit roll-out in the West Midlands}
%
%\href{https://www.birminghammail.co.uk/black-country/universal-credit-claimants-black-country-15946831}{Universal credit roll-out in the Black Country}
%
%
%food bank dependence? data on housing e.g. evictions? homelessness.
%
%Mention we can construct some of these variables by access to "a very large UK bank" which whom we have a data sharing agreement. We cna build them from the row-per-transactino data for spending on current account and credit cards.

\newpage

\bibliographystyle{apacite}
\bibliography{domesticabuse_refs}

\end{document}
